\chapter{Conclusions}
\label{chapter:conclusion}
This thesis set out to investigate the performance of OpenEM based stream processing applications, with specific focus on scheduling. Scheduling is an important factor of the streaming performance. A good scheduler is capable of utilizing parallelism to improve throughput and latency of the application, whereas worse schedulers do not distribute work on multiple cores as efficiently and thus execute more of the work sequentially leading to increased latency and decreased throughput.

The problem was approached by comparing the performance of an OpenEM based streaming application to the performance of a similar, statically scheduled application. The workload application designed for the experiments is a video stream processing application inspired by the Canny edge detector. The baseline application was implemented using PREESM, which generates statically scheduled multi-core applications based on synchronous dataflow graphs. The applications were instrumented so that they could be compared in terms of latency, throughput, and core utilization. The results of the comparison experiment suggest that the OpenEM dynamic scheduler does not cause unmanageable overhead compared to statically scheduled applications.

The scheduling performance of OpenEM was further investigated with two experiments carried out using the same measurement system. In one experiment the balance between throughput and latency was examined by increasing the number of frames processed concurrently. This experiment showed that at least simple OpenEM applications are adjustable in terms of throughput and latency, which may be useful in stream processing tasks. In the other experiment the parallelizing performance of the scheduler was examined by computing the same tasks using variable number of cores. The growth in performance was 90\% of linear growth from one to eight cores, which is decent improvement considering that in addition to the parallel actors, the workload had sequential actors, which did not benefit from the parallelization.

The results of the experiments suggest that Texas Instruments OpenEM runtime is flexible enough to support the implementation of streaming applications and it was found that the dynamic scheduler is capable of making reasonable scheduling decisions with small overhead. Further research is required to understand the overall performance of multi-core DSPs versus GPUs, FPGAs, and CPUs in stream processing.

As data streaming is becoming more widespread in the Internet, high performance stream processing solutions are in demand in the industry. Exploring the use of DSPs for stream processing is interesting because they achieve high ratio of floating point operations to energy consumption and they implement an architecture that is well suited to stream processing in theory. As multi-core DSPs are not as commonly used for stream processing as for example GPUs, the toolset for parallel programming is not as standardized as on other platforms. Therefore, runtime systems such as OpenEM have the potential of becoming the standard tools for stream processing on DSPs.
