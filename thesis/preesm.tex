\chapter{PREESM}
\label{chapter:preesm}
In this chapter the PREESM rapid prototyping framework is introduced. First an
overview of the framework is given. Second Dataflow models of computation are
discussed. Third development using the PREESM framework is discussed.

\section{PREESM Overview}
\label{sec:preesmover}
The PREESM rapid prototyping framework for multicore development was used to
create a workload application for the thesis experiment \ref{firstexperiment}.
PREESM applications are built using dataflow model. PREESM provides graphical
tools for editing the dataflow model and it generates multicore code that is
guaranteed to be deadlock free \cite{pelcat2014preesm}. 

The reason PREESM was used in the experiment was the need for a simple way to
create multicore applications to provide a baseline for the OpenEM application.

\section{Dataflow Models of Computation}
\label{sec:dataflow}

\section{Development with PREESM}
\label{sec:preesmdev}
