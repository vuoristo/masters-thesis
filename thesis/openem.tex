\chapter{Open Event Machine}
\label{chapter:openem}
This thesis investigates the suitability of Open Event Machine (OpenEM) for
real-time stream processing. OpenEM is a programming framework for event-driven
multicore applications.

\section{OpenEM Framework}
\subsection{Overview}
OpenEM is an event-driven programming framework originally developed for the
networking data plane by Nokia Solutions and Networks. The OpenEM framework
provides a programming model for scalable and dynamically load balanced
applications. The key components of the OpenEM programming model are events,
execution objects, queues and the scheduler. OpenEM works with
run-to-completion principle which means once an event begins to execute it will
not be interrupted by the runtime. The run-to-completion principle implies
limitations within which well performing applications must be designed, the
main limitation being the required small granule size for computations.
\cite{openempage}

The unit of communication in OpenEM is the Event described in
\ref{subsec:event}. Events are sent to Queues (\ref{subsec:queues}). Queues are
connected to Execution Objects (\ref{subsec:eos}). The scheduler
(\ref{subsec:schedule}) chooses an event from a suitable queue based on
scheduling rules and schedules it on a core.
\subsection{Events}
\label{subsec:event}
\subsection{Execution Objects}
\label{subsec:eos}
\subsection{Queues}
\label{subsec:queues}
\subsection{Scheduling}
\label{subsec:schedule}

\section{Texas Instruments Implementation of OpenEM}
\subsection{Multicore Navigator and OpenEM}
\subsection{State of TI OpenEM Implementation}


