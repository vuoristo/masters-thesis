\chapter{Introduction}
\label{chapter:introduction}
Digital data is being created and copied over the Internet at an accelerating pace. A large fraction of this data is transferred and processed in the form of data streams such as video. In general streaming data is processed as it flows by the processing node, without saving a copy of the complete stream on the node. For example a video streaming application buffers some number of video frames on the viewing device, decodes and presents them and then frees the memory for buffering more frames.

Stream processing programs consist of series of operations that are performed in parallel on the streams. The capability for parallelizing the programs is important for creating low-latency, high-throughput applications. In the past single-threaded performance of computers was growing rapidly, which made hitting the latency and throughput constraints of the applications a question of waiting for the next generation CPUs, which would make all of the applications faster. For the past ten years or so, the single-threaded performance has remained roughly the same while the number of cores in CPUs has grown. This turn towards parallelism in hardware has emphasised the importance of parallel processing in streaming applications.

Traditional CPUs are not the only multicore devices capable of stream processing. Different hardware architectures implemented by digital signal processors (DSP) and graphics processing units (GPU) have been successfully used for various forms of parallel computing. Each of the hardware architectures offer their own set of advantages: CPUs have highly optimized architecture for processing full-featured threads, GPUs excel in data parallelism and DSPs offer low energy consumption in signal processing applications. DSPs have the potential for becoming an efficient platform for stream processing, as signal processing applications have similar structure to stream processing applications.

There exists a variety of forms of parallel processing including task parallelism. In task parallelism units of work called tasks consisting of code and data are distributed on the cores of the computer. Stream processing applications can be seen from the task parallel perspective with each of the operations performed on the streams making up a separate task. In this thesis the performance of a task based programming model in stream processing on a multicore DSP is investigated. The task based programming model under study is the Open Event Machine (OpenEM).

Important aspect of the performance of task parallel programming models is how the tasks are scheduled on the cores. OpenEM features a dynamic scheduler, which means the execution order of tasks is decided in runtime. Another way of scheduling tasks would be static scheduling, where the execution order of the tasks is determined at the time of compilation. To put the OpenEM performance in perspective, its performance is compared to the performance of statically scheduled applications. The PREESM framework is used for creating the statically scheduled application. PREESM is a tool for creating DSP statically scheduled parallel applications following the synchronous dataflow model of computation.

The hardware platform used for the experiments conducted in this thesis is the Texas Instruments TMS320C6678 multicore DSP. Out of other possible multicore DSPs the particular device was selected for the experiments conducted in this thesis most importantly because Texas Instrument provides an implementation of OpenEM for the device. Another advantage of the TMS320C6678 is that PREESM supports the processor as a target for code generation, which makes creating comparable applications easier.

Two stream processing applications are implemented: one using OpenEM and the other created with PREESM, using static scheduling. The main contribution of this thesis is the analysis of the performance of OpenEM in stream processing tasks.

The relevancy of studying OpenEM on multicore digital signal processors for stream processing comes from a combination of factors. First, stream processing is a computing paradigm that is receiving increasing attention from the industry and the research community due to the growing volume of data streaming in the Internet. Second, stream processing tasks have similar high-level structure as digital signal processing tasks. Examining the fit of digital signal processors for stream processing is thus a task relevant to the study of stream processing. Third, multicore coordination is non-trivial but parallel processing is required for efficient stream processing as stream processing tasks are highly data parallel. The dynamic scheduler implemented in OpenEM has large impact on the performance of the stream processing applications using it.

\section{Problem statement}
\label{section:problem-statement}
Multicore DSPs have potential for high performance stream processing in terms of floating point operations per Joule. However, parallel programming for multicore DSPs is more complicated than parallel programming for PC hardware, as commonly there is no operating system to handle the inter-core coordination and communication. The TI implementation of OpenEM is a runtime system that handles the inter-core coordination and communication. This thesis investigates the performance of the TI OpenEM implementation in stream processing.

The performance of a multicore runtime system is dependent on the efficiency of the scheduling. TI OpenEM implements dynamic scheduling. The scheduler is hardware accelerated, running on a special processor that is connected to the hardware queuing system of the processor. The communication in TI OpenEM utilises the hardware queues of the processor.

The dynamic scheduler can negatively affect the performance of the application versus a statically scheduled application if the scheduling requires a large overhead in terms of processor cycles or memory. In addition to the efficiency of the dynamic scheduler, the overall goodness of the schedule affects the application performance. If the scheduler (static or dynamic) places all the tasks sequentially, none of the benefits of parallelization are achieved. Another more subtle effect the schedule has on the performance is the memory locality of the tasks. If tasks operating on the same data are executed on different processors or other tasks are run on the same processor between the tasks, the data has to be copied between the processors, whereas if the tasks execute sequentially on the same processor, no time is wasted in copying the data.

Three views to the performance of OpenEM in stream processing are taken through three different experiments. The performance of the dynamic scheduler is assessed by comparing an application implemented using OpenEM to a statically scheduled application implemented with PREESM in experiment~\ref{subsec:first-experiment}. The capabilities of the dynamic scheduler are further investigated by looking at the balance of throughput and latency in experiment~\ref{subsec:second-experiment}. Finally, an understanding of the parallelizing performance of the scheduler is sought out in the experiment~\ref{subsec:third-experiment} where the stream processing application is run on different numbers of cores and the results are compared.

\section{Contributions}
\label{section:contributions}
This thesis presents analysis of the performance of OpenEM in stream processing. The performance analysis is carried out by comparing the measured performance of a stream processing application implemented using OpenEM to a comparable stream processing application.

The application used for comparison was implemented using dataflow model of computation with a graphical tool called PREESM. The PREESM tool is able to generate executables for the Texas Instruments TMS320C6678 DSP from the graphical model of the dataflow application and source code for the processing kernels provided by the user. The dataflow graph, the processing kernels and the instrumentation of the PREESM application were implemented for this thesis.

The main focus in the experiments is on the the OpenEM based stream processing application. The application was implemented from scratch for this thesis. The Texas Instruments Code Composer Studio was used to implement the stream processing application. The application was instrumented using the hardware performance counters found in the TI DSP.

In the analysis phase, the performance of the applications was compared based on the measurements made using the instrumentation. Three experiments were conducted to understand the performance of OpenEM based applications in stream processing.

The contributions of this thesis are summarized in the following listing:
\begin{itemize}
    \item Implementation and instrumentation of an OpenEM based measurement system on TI DSP.
    \item Implementation and instrumentation of a comparable measurement system using PREESM.
    \item Analysis of the OpenEM scheduler performance in scheduling a stream processing application is conducted by comparing OpenEM dynamic scheduling with static scheduling on TI DSP.
    \item Examination of the adjustability of the OpenEM scheduler in terms of latency versus throughput is carried out.
    \item Analysis of the efficiency of the OpenEM scheduler in parallel processing is provided.
\end{itemize}

\section{Structure of the Thesis}
\label{section:structure}
The structure of this thesis is presented in the following.

Chapter~\ref{chapter:streams} introduces the context of this thesis. The relevance of stream processing to computer science research and the industry are explained. An overview to stream processing is given and a more focused look at one paradigm of stream processing, dataflow, is taken.

Chapter~\ref{chapter:openem} takes an in-depth look at Open Event Machine. First, the context of parallel computing is introduced and the forms of parallelism implemented in OpenEM are described. Second, the Nokia Solutions and Networks specification of the framework is introduced. Third the Texas Instruments implementation of OpenEM for TMS320C6678 multicore DSP is examined.

Chapter~\ref{chapter:experiments} introduces the material and methods used in the experiments. The chapter starts by looking at the performance analysis of computer systems. More specifically measuring computer systems is described. Next the hardware platform used in the experiments, the TI TMS320C6678 is described. Following the hardware description, PREESM is introduced as the tool that was used for creating the comparison point for the OpenEM stream processing application. After PREESM, video stream standards relevant to the experiments are explained. Finally, Canny edge detector that served as inspiration for the workload is described.

Chapter~\ref{chapter:construction} describes the construction of the experiments. After the introduction to the experiments the experiment workload is introduced. Next the PREESM version and the OpenEM version of the workload application are described. Third the instrumentation of the applications is explained. Finally the experiments are presented.

Chapter~\ref{chapter:results-and-analysis} presents the results of the experiments. In the beginning of the chapter the results are given and described. The rest of the chapter is dedicated for discussion of the results.

Chapter~\ref{chapter:conclusion} is the conclusion of this thesis, bringing the results to the broader context of stream processing and parallel computing.
