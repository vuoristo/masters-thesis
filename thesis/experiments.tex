\chapter [OpenEM Experiments] {Characterizing Open Event Machine Performance
Through Experiments}
\label{chapter:experiments}
The objective of these experiments is to understand the behavior and
performance of the Texas Instruments implementation of Open Event Machine in
realtime stream processing. The objective is achieved by comparing the behavior
of an application implemented using OpenEM to the behavior of a similar
application implemented using a simpler multicore runtime system (PREESM) in the
first part of the experiment. The second part of the experiment is the
construction of a simulation model. The performance predictions from the
simulation model will be compared to the performance of the real world
application. The objective of the comparison is to help better understand the
OpenEM platform.

Both of the experiments are described in a similar manner. First the parameters
and factors of the experiments are explained, second the different measurement
setups are introduced and third the results of the measurements are presented.
The comparison of the PREESM and OpenEM applications is described in the section
\ref{sec:firstexperiment}. The performance of the simulation model is described
in \ref{sec:secondexperiment}.

\section{Comparison of PREESM and OpenEM Filter Applications}
\label{sec:firstexperiment}
In the first experiment the OpenEM and PREESM filter applications were loaded
with similar workloads and their execution was measured. The idea of the first
experiment is to examine the dynamic scheduling capabilities of the OpenEM
scheduler and the overhead of the OpenEM framework in stream processing. The
OpenEM scheduler is hardware accelerated, running on a separate processor in the
TMS320C6678 chip. The scheduling is explained in detail in chapter
\ref{chapter:openem}. To achieve this objective the OpenEM filter application
introduced in chapter \ref{chapter:construction} was measured under different
loads and compared to a similar application implemented using PREESM. The
applications are not comparable in terms of throughput and latency, because the
runtime systems are designed for different purposes. The PREESM application
should be considered a baseline, which demonstrates how a statically scheduled
application behaves under dynamic workload.

The static schedule of the PREESM application was regenerated between every
measurement setup due to the limitations of the code generation in PREESM
framework. The specific limitation was that the parameters of the actor model
were translated to static memory allocations in the code generator, and manually
changing the generate allocations would've been complicated and prone to error.
As a result of this the actors are scheduled slightly differently between each
scenario. To demonstrate the effect of static scheduling, the estimated actor
timings of the PREESM application were not modified when changing the frame
size.

In this experiment the applications are loaded with three different workloads
and measured. In addition the OpenEM application is measured under the same load
but different numbers of available cores. The experiment is explained in the
following subsections. In the first subsection the parameters and factors of the
experiment are introduced. Second the different measurement setups are described
and third the results of the experiment are presented.

\subsection{Parameters and Factors}
Dynamic workload conditions are emulated by repeating the measurements with
different factors. To keep things simple the video streams are not dynamically
switched at runtime. The measurement parameters are presented in the following
listing.

\begin{itemize}
    \item \textbf{Video Frame Size} - The workloads are differentiated by
        changing the frame sizes of the video streams.
    \item \textbf{OpenEM Core Masks} - The OpenEM application is measured
        with different core masks of the Execution Objects.
    \item \textbf{Number of Frames Processed Simultaneously} - The OpenEM
        application processes variable number of frames simultaneously, which
        affects the latency and throughput of the application.
\end{itemize}

The different video frame sizes used are presented in table
\ref{tab:cif_frames}. The frame sizes used are selected from among the Common
Intermediate Format frame sizes. TODO: find a reference for CIF. YUV video
format is used in the applications, but only the Y channel is processed by the
applications. The Y channel in the YUV frame contains $R_{x} * R_{y}$ bytes
where $R$ is the resolution. In the YUV format used the U and V channels have
reduced bitrates of $\frac{1}{4} * R_{x} * R_{y}$ per frame.

\begin{table}
    \begin{center}
        \begin{tabular}{ c c c }
            Name  & X resolution  & Y resolution \\ \hline
            QCIF  & 176           & 144          \\ \hline
            CIF   & 352           & 288          \\ \hline
            4CIF  & 704           & 576          \\ \hline
        \end{tabular}
        \caption{CIF frame sizes}
        \label{tab:cif_frames}
    \end{center}
\end{table}

In the second part of this experiment OpenEM core masks are used to limit the
number of cores available to the filter application. The behavior of OpenEM is
examined under limited resources. The core masks in Texas Instruments
implementation of OpenEM are limited so that only one core mask can be active in
the application as discussed in \ref{chapter:openem}, therefore the core masks
always apply to all execution objects of the application.

\subsection{Measurement Setups}
The filter applications process two video streams simultaneously as described in
chapter \ref{chapter:construction}. One video stream is processed with a sobel
filter and the other is processed with a gaussian filter. The dynamic behavior
of the applications is investigated using different workloads. The workloads
used are presented in the table \ref{tab:preesm_setups}. The purpose of the
different bitrates used for each video stream is to expose the behavior of the
OpenEM scheduler in handling dynamic workloads. The static schedule in the
PREESM application will provide a baseline to reflect the OpenEM performance to,
but again the performance of the applications should not be directly compared
due to the difference in the runtime systems.

In addition to comparing the PREESM and OpenEM applications the OpenEM
application is measured with different core masks to investigate the dynamic
scheduling with different limitations. Both filters of the OpenEM application
are loaded with CIF streams and different numbers of cores are used. The
experiment is run with core masks allowing one to eight cores being used for
processing the streams.

\begin{table}
    \begin{center}
        \begin{tabular}{ c c }
            Sobel Resolution & Gauss Resolution \\ \hline
            CIF              & CIF              \\ \hline
            4CIF             & CIF              \\ \hline
            CIF              & 4CIF             \\ \hline
            QCIF             & QCIF             \\ \hline
        \end{tabular}
        \caption{PREESM and OpenEM measurement setups}
        \label{tab:preesm_setups}
    \end{center}
\end{table}

\subsection{Results}
In this subsection first the results of the measurements of the OpenEM and
PREESM applications are presented and next the results of measuring the OpenEM
application limited with core masks are presented. The latency and throughput of
the OpenEM application measurements are summarized in the table
\ref{tab:oemthrough} and in table \ref{tab:preesmthrough} of the PREESM
application. The summary of latency and throughtput of the OpenEM application
with limited number of cores is presented in table \ref{tab:oemcoremasks}.

The latency of both of the filters is measured from the time the frame is loaded
from the memory to the time the frame is merged after filtering. The throughput
is measured as frames per second processed in total by the application. Since
both of the applications process frames at the same rate from both streams,
there is only one FPS measurement for each of the measurement setups.

The PREESM latencies in the table \ref{tab:preesmthrough} are consistently
smaller than the latencies of the OpenEM application in the table
\ref{tab:oemthrough}. This corresponds to the difference of the block schedule
used in the PREESM application to the dynamic schedule used in the OpenEM
application. The OpenEM schedule is more optimized for throughput which is
readily observed from the difference of the FPS metrics for both applications.
The throughput versus latency balance in the OpenEM application is controlled
through the number of frames processed simultaneously. In this set of
measurements the OpenEM application is configured to process 16 frames
simultaneously to maximize throughput. \\

TODO: Add another set of OpenEM measurements using smaller number of initial
events for better latency.\\

\newcommand{\head}[2]{\multicolumn{1}{>{\centering\arraybackslash}p{#1}}{#2}}
\begin{table}
    \begin{center}
        \begin{tabular}{ c c c c c }
            \head{1.5cm}{Sobel latency} & \head{1.5cm}{Gauss latency} &
            \head{1.5cm}{FPS} & \head{1.5cm}{Sobel frame size} &
            \head{1.5cm}{Gauss frame size} \\ \hline
            5,41 & 8,78 & 223 & CIF & 4CIF \\ \hline
            4,65 & 3,54 & 334 & 4CIF & CIF \\ \hline
            2,15 & 2,51 & 668 & CIF & CIF \\ \hline
            0,61 & 0,71 & 2004 & QCIF & QCIF \\ \hline
        \end{tabular}
        \caption{PREESM latency and throughput. The the latencies are measured
        in milliseconds.}
        \label{tab:preesmthrough}
    \end{center}
\end{table}

\begin{table}
    \begin{center}
        \begin{tabular}{ c c c c c }
            \head{1.5cm}{Sobel latency} & \head{1.5cm}{Gauss latency} &
            \head{1.5cm}{FPS} & \head{1.5cm}{Sobel frame size} &
            \head{1.5cm}{Gauss frame size} \\ \hline
            15,82 & 22,85 & 599 & CIF & 4CIF \\ \hline
            4,85 & 3,67 & 895 & 4CIF & CIF \\ \hline
            4,91 & 5,96 & 1955 & CIF & CIF \\ \hline
            1,33 & 1,62 & 7819 & QCIF & QCIF \\ \hline
        \end{tabular}
        \caption{OpenEM latency and throughput. The latencies are measured in
        milliseconds.}
        \label{tab:oemthrough}
    \end{center}
\end{table}

A block schedule could be generated, which would yield a higher throughput by
introducing multiple instances of the actors for each repetition of the
schedule. The successive repetitions of the schedule would be interleaved and
the overhead of synchronizing the cores would be reduced. The current version of
PREESM does not support the described interleaving of the repetitions
\cite{pelcat2014preesm}.

The throughput of the application is determined by how much time the application
spends processing the streams versus the time spent doing something else. For
example the synchronization of all cores before each repetition of the PREESM
schedule consumes a lot of cpu cycles as is readily observed from the figure
\ref{fig:preesmcif}. The portion of the bars marked as busy corresponds to the
cycles spent in the synchronization between the repetitions of the schedule. The
percentage of cycles spent in the synchronization varies from 16\% on Core 7
to 45\% on Core 0. Comparing the core utilization of the PREESM application to
the core utilization of the OpenEM application presented in the figure
\ref{fig:oem8corefunc} the differences in the schedulers start to become apparent.

One to one comparison of the cpu utilization of the applications is difficult
due to the differences in the runtime systems. The overhead portions in the
figures \ref{fig:preesmcif} and \ref{fig:oem8corefunc} contain all of the data
copying from buffer to buffer outside the measured functions, but they also
contain different amounts of cycles spent in communications between the cores. A
more revealing view at the overhead of the OpenEM runtime is found by looking at
the cycles spent per execution object in the OpenEM application in figure
\ref{fig:oem8coreeo}, where the overhead corresponds to cycles spent outside the
execution objects. The cycles spent outside the execution objects are maximum of
3\% on all cores.

The throughput of both of the applications scales roughly in proportion to each
other across the different workloads in tables \ref{tab:preesmthrough} and
\ref{tab:oemthrough}. However the latencies of the OpenEM application show more
variable behavior. The latencies of both applications are roughly the same in
the case of two CIF streams and are improved roughly by a factor of 3.5 when the
bitrates are decreased in the case of two QCIF streams. The latencies of the
PREESM application grow steadily as the load of the application is increased.
The latencies of the OpenEM application actually improve when the resolution of
the sobel stream is increased to 4CIF and then drastically grow in the heaviest
workload case with sobel resolution of CIF and gauss resolution of 4CIF.

The probable cause of the improvement of the latencies when the workload is made
heavier by increasing the sobel stream resolution is the reduced interleaving of
the processing of the subsequent frames. The reduction in the interleaving is
caused by the read execution object which is only connected to an atomic queue.
When the frame size of the sobel stream is increased the read EO starts limiting
the throughput of the application. Fewer frames are processed in parallel which
decreases the time from reading each individual frame to the completion of that
frame. The formation of the bottleneck can be observed by comparing the core
utilization in the figure \ref{fig:oem8coreeo} to the core utilization in the
figure \ref{fig:oem8coreeosobel4cif} where seven cores need to wait for the read
operations on the Core 2 and the overall overhead is increased. The other cores
do not receive any events from the OpenEM scheduler while waiting.

When the gauss stream is increased to 4CIF and the sobel stream is kept at CIF
the bottleneck does not form because computing the gaussian filter for the 4CIF
frames is consuming approximately 80\% of the cycles on all cores. In this case
the read EO only consumes approximately 5\% to 13\% of the cycles on all cores.
Compared to the case of two CIF streams, the latencies in this case are
increased by factors of approximately 3 and 4 for sobel and gauss
correspondingly. The resulting core utilization graph is presented in figure
\ref{fig:oem8coreeogauss4cif}.

\begin{figure}
    \begin{center}
        \includegraphics[width=0.99\textwidth]{images/preesm_cifcif.eps}
        \captionof{figure}{PREESM cycles spent per function for CIF sobel frames
            and CIF gauss frames. The busy portion of the bars correspond to the
            cycles spent synchronizing the cores between the repetitions of the
            block schedule. The overhead corresponds to cycles spent outside the
            measured functions and the busy cycles.} \label{fig:preesmcif}
    \end{center}
\end{figure}

\begin{figure}
    \begin{center}
        \includegraphics[width=0.99\textwidth]{images/openem_cifcif_8cores_func.eps}
        \captionof{figure}{OpenEM cycles spent per function for CIF sobel frames
        and CIF gauss frames. The overhead corresponds to cycles spent outside
        the listed functions, which includes the communication overhead and the
        copying of the data between the buffers.}
        \label{fig:oem8corefunc}
    \end{center}
\end{figure}

\begin{figure}[h!]
    \begin{center}
        \includegraphics[width=0.99\textwidth]{images/openem_cifcif_8cores_eo.eps}
        \caption{OpenEM cycles spent per execution object for CIF sobel frames
        and CIF gauss frames. The overhead corresponds to cycles spent outside
        the execution objects.}
        \label{fig:oem8coreeo}
    \end{center}
\end{figure}

\begin{figure}[h!]
    \begin{center}
        \includegraphics[width=0.99\textwidth]{images/openem_sobel4cif_gausscif_eo.eps}
        \caption{OpenEM cycles spent per execution object for 4CIF sobel frames
        and CIF gauss frames. Core 2 only executes the read EO and other cores
        have to wait for the read operations to complete which results in
        increased overhead.}
        \label{fig:oem8coreeosobel4cif}
    \end{center}
\end{figure}

\begin{figure}[h!]
    \begin{center}
        \includegraphics[width=0.99\textwidth]{images/openem_sobelcif_gauss4cif_eo.eps}
        \caption{OpenEM cycles spent per execution object for CIF sobel frames
        and 4CIF gauss frames}
        \label{fig:oem8coreeogauss4cif}
    \end{center}
\end{figure}

The second part of this experiment was to limit the number of cores available
for the OpenEM application. The resulting latencies and throughputs are
presented in the table \ref{tab:oemcoremasks}. The increase of the throughput of
the OpenEM application is presented in graph \ref{fig:fpsvcores}. The actual FPS
measured with eight cores is approximately 90\% of the linear growth.

\begin{table}
    \begin{center}
        \begin{tabular}{ c c c c }
            \head{1.5cm}{Sobel latency} & \head{1.5cm}{Gauss latency} &
            \head{1.5cm}{FPS} & \head{1.5cm}{Number of cores} \\
            \hline
            57,05 & 57,11 & 263 & 1 \\ \hline
            22,59 & 23,15 & 510 & 2 \\ \hline
            15,09 & 15,84 & 768 & 3 \\ \hline
            10,91 & 11,85 & 1014 & 4 \\ \hline
            8,41 & 9,53 & 1268 & 5 \\ \hline
            7,05 & 8,07 & 1500 & 6 \\ \hline
            5,74 & 6,83 & 1731 & 7 \\ \hline
            4,91 & 5,96 & 1955 & 8 \\ \hline
        \end{tabular}
        \caption{OpenEM measurements with number of cores varied}
        \label{tab:oemcoremasks}
    \end{center}
\end{table}

\begin{figure}[h!]
    \begin{center}
        \includegraphics[width=0.99\textwidth]{images/coremask_fps.eps}
        \caption{FPS increase as the function of cores vs. linear growth}
        \label{fig:fpsvcores}
    \end{center}
\end{figure}

\section{Comparison of Simulated OpenEM Performance and Real OpenEM Performance}
\label{sec:secondexperiment}
\subsection{Parameters and Factors}
\subsection{Measurement Setups}
\subsection{Results}

