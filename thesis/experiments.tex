\chapter{Characterizing Open Event Machine Performance Through Experiments}
\label{chapter:experiments}
The objective of these experiments is to understand the behavior and
performance of the Texas Instruments implementation of Open Event Machine in
realtime stream processing. The objective is achieved by comparing the behavior
of an application implemented using OpenEM to the behavior of a similar
application implemented using a simpler multicore runtime system (PREESM) in the
first part of the experiment. The second part of the experiment is the
construction of a simulation model. The performance predictions from the
simulation model will be compared to the performance of the real world
application. The objective of the comparison is to help better understand the
OpenEM platform.

Both of the experiments are described in a similar manner. First the parameters
and factors of the experiments are explained, second the different measurement
setups are introduced and third the results of the measurements are presented.
The comparison of the PREESM and OpenEM applications is described in the section
\ref{sec:firstexperiment}. The performance of the simulation model is described
in \ref{sec:secondexperiment}.

\section{Comparison of PREESM and OpenEM Filter Applications}
\label{sec:firstexperiment}
In the first experiment the OpenEM and PREESM filter applications were loaded
with similar workloads and their execution was measured.
\subsection{Parameters and Factors}
\subsection{Measurement Setups}
\subsection{Results}
\section{Comparison of Simulated OpenEM Performance and Real OpenEM Performance}
\label{sec:secondexperiment}
\subsection{Parameters and Factors}
\subsection{Measurement Setups}
\subsection{Results}

