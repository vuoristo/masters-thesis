\chapter [OpenEM Experiments] {Characterizing Open Event Machine Performance
Through Experiments}
\label{chapter:experiments}
The objective of these experiments is to understand the behavior and
performance of the Texas Instruments implementation of Open Event Machine in
realtime stream processing. The objective is achieved by comparing the behavior
of an application implemented using OpenEM to the behavior of a similar
application implemented using a simpler multicore runtime system (PREESM) in the
first part of the experiment. The second part of the experiment is the
construction of a simulation model. The performance predictions from the
simulation model will be compared to the performance of the real world
application. The objective of the comparison is to help better understand the
OpenEM platform.

Both of the experiments are described in a similar manner. First the parameters
and factors of the experiments are explained, second the different measurement
setups are introduced and third the results of the measurements are presented.
The comparison of the PREESM and OpenEM applications is described in the section
\ref{sec:firstexperiment}. The performance of the simulation model is described
in \ref{sec:secondexperiment}.

\section{Comparison of PREESM and OpenEM Filter Applications}
\label{sec:firstexperiment}
In the first experiment the OpenEM and PREESM filter applications were loaded
with similar workloads and their execution was measured. The idea of the first
experiment is to examine the dynamic scheduling capabilities of the OpenEM
scheduler and the overhead of the OpenEM framework in stream processing. The
OpenEM scheduler is hardware accelerated, running on a separate processor in the
TMS320C6678 chip. The scheduling is explained in detail in chapter
\ref{chapter:openem}. To achieve this objective the OpenEM filter application
introduced in chapter \ref{chapter:construction} was measured under different
loads and compared to a similar application implemented using PREESM. The
applications are not comparable in terms of throughput and latency, because the
runtime systems are designed for different purposes. The PREESM application
should be considered a baseline, which demonstrates how a statically scheduled
application behaves under dynamic workload.

The static schedule of the PREESM application was regenerated between every
measurement setup due to the limitations of the code generation in PREESM
framework. The specific limitation was that the parameters of the actor model
were translated to static memory allocations in the code generator, and manually
changing the generate allocations would've been complicated and prone to error.
As a result of this the actors are scheduled slightly differently between each
scenario. To demonstrate the effect of static scheduling, the estimated actor
timings of the PREESM application were not modified when changing the frame
size.

In this experiment the applications are loaded with three different workloads
and measured. In addition the OpenEM application is measured under the same load
but different numbers of available cores. The experiment is explained in the
following subsections. In the first subsection the parameters and factors of the
experiment are introduced. Second the different measurement setups are described
and third the results of the experiment are presented.

\subsection{Parameters and Factors}
Dynamic workload conditions are emulated by repeating the measurements with
different factors. To keep things simple the video streams are not dynamically
switched at runtime. The measurement parameters are presented in the following
listing.

\begin{itemize}
    \item \textbf{Video Frame Size} - The workloads are differentiated by
        changing the frame sizes of the video streams.
    \item \textbf{OpenEM Core Masks} - The OpenEM application is measured
        with different core masks of the Execution Objects.
\end{itemize}

\subsection{Measurement Setups}
The the configurations of the OpenEM and PREESM applications for the different
measurement setups are presented here. The frame sizes used are from the CIF
standard TODO: reference. The CIF frame sizes are presented in the following
table.

\begin{table}
    \begin{center}
        \begin{tabular}{ c c c }
            Name  & x resolution  & y resolution \\ \hline
            QCIF  & 176           & 144          \\ \hline
            CIF   & 352           & 288          \\ \hline
            4CIF  & 704           & 576          \\ \hline
        \end{tabular}
        \caption{CIF frame sizes}
        \label{tab:cif_frames}
    \end{center}
\end{table}

The applications were measured in setups presented in table
\ref{tab:preesm_setups}. In addition to the measurement setups presented in the
table, the OpenEM application was measured with one to eight cores at use at
fime with both streams at CIF resolution.

\begin{table}
    \begin{center}
        \begin{tabular}{ c c }
            Sobel Resolution & Gauss Resolution \\ \hline
            CIF              & CIF              \\ \hline
            4CIF             & CIF              \\ \hline
            CIF              & 4CIF             \\ \hline
            QCIF             & QCIF             \\ \hline
        \end{tabular}
        \caption{PREESM and OpenEM measurement setups}
        \label{tab:preesm_setups}
    \end{center}
\end{table}

\newcommand{\head}[2]{\multicolumn{1}{>{\centering\arraybackslash}p{#1}}{#2}}

\subsection{Results}
\begin{table}
    \begin{center}
        \begin{tabular}{ c c c c }
            \head{1.5cm}{Sobel latency} & \head{1.5cm}{Gauss latency} &
            \head{1.5cm}{FPS} & \head{1.5cm}{Number of cores} \\
            \hline
            57,05 & 57,11 & 263 & 1 \\ \hline
            22,59 & 23,15 & 510 & 2 \\ \hline
            15,09 & 15,84 & 768 & 3 \\ \hline
            10,91 & 11,85 & 1014 & 4 \\ \hline
            8,41 & 9,53 & 1268 & 5 \\ \hline
            7,05 & 8,07 & 1500 & 6 \\ \hline
            5,74 & 6,83 & 1731 & 7 \\ \hline
            4,91 & 5,96 & 1955 & 8 \\ \hline
        \end{tabular}
        \caption{OpenEM measurements with number of cores varied}
        \label{tab:m_}
    \end{center}
\end{table}

\begin{table}
    \begin{center}
        \begin{tabular}{ c c c c c }
            \head{1.5cm}{Sobel latency} & \head{1.5cm}{Gauss latency} &
            \head{1.5cm}{FPS} & \head{1.5cm}{Sobel frame size} &
            \head{1.5cm}{Gauss frame size} \\ \hline
            15,82 & 22,85 & 599 & CIF & 4CIF \\ \hline
            4,85 & 3,67 & 895 & 4CIF & CIF \\ \hline
            4,91 & 5,96 & 1955 & CIF & CIF \\ \hline
            1,33 & 1,62 & 7819 & QCIF & QCIF \\ \hline
        \end{tabular}
        \caption{OpenEM latency and throughput}
    \end{center}
\end{table}

\begin{table}
    \begin{center}
        \begin{tabular}{ c c c c c }
            \head{1.5cm}{Sobel latency} & \head{1.5cm}{Gauss latency} &
            \head{1.5cm}{FPS} & \head{1.5cm}{Sobel frame size} &
            \head{1.5cm}{Gauss frame size} \\ \hline
            5,41 & 8,78 & 223 & CIF & 4CIF \\ \hline
            4,65 & 3,54 & 334 & 4CIF & CIF \\ \hline
            2,15 & 2,51 & 668 & CIF & CIF \\ \hline
            0,61 & 0,71 & 2004 & QCIF & QCIF \\ \hline
        \end{tabular}
        \caption{PREESM latency and throughput}
    \end{center}
\end{table}

\section{Comparison of Simulated OpenEM Performance and Real OpenEM Performance}
\label{sec:secondexperiment}
\subsection{Parameters and Factors}
\subsection{Measurement Setups}
\subsection{Results}

