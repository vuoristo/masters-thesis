\chapter[Texas Instruments TMS320C6678]{Texas Instruments\\TMS320C6678}
\label{chapter:c6678}
This chapter describes the Texas Instruments TMS320C6678 multicore DSP. First
the selection of the TMS320C6678 as the hardware platform for the experiments
is explained. Second the key hardware features of the platform and
the related development tools are examined.

\section{Selection of the Hardware Platform}
The hardware platform used in the experiments in this thesis is the  
Advantech TMDSEVM6678L TMS320C6678 Evaluation Module. The processor in the
evaluation module is the Texas Instruments TMS320C6678. The TMS320C6678
is a fixed and floating point digital signal processor based on the
Texas Instruments Keystone I architecture. The processor has eight
C66x DSP cores at 1.0 GHz clock frequency \cite{tmsdatasheet}.

The TMS320C6678 was chosen as the hardware platform for the experiments
for two main reasons. First the Keystone I architecture is designed with
stream processing applications in mind \cite{multicorevideo}. Second the
objective of this thesis is to understand hardware and software features
supported by the TMS320C6678.

The specific features in the TMS320C6678 we are interested in are the
support for OpenEM included in the MCSDK for Keystone I devices
\cite{MCSDKbrochure} and the hardware accelerated scheduling and
communication utilized by the OpenEM runtime. 

An additional benefit of choosing TMS320C6678 as the hardware platform for
the experiments is the support for the platform in PREESM
\cite{pelcat2014preesm}.
\section{TMS320C6678 Overview}

\begin{figure}[h!]
    \label{arch_overview}
    \begin{center}
        \includegraphics[width=0.99\textwidth]{images/fbd_SPRS691e.png} 
        \caption{Overview of the TMS320C6678 architecture.}
    \end{center}
\end{figure}

The Keystone I architecture the TMS320C6678 is based on specifies a set
of hardware elements which enable integration of DSP cores, application 
specific co-processors and IO \cite{tmsdatasheet}. In the Keystone I 
architecture there are multiple ways for the C66x cores to communicate
with each other, the memory and the peripherals. The methods of
communication of specific interest for the experiments in this thesis
are communication through shared memory and communication through
Multicore Navigator.

Keystone I architecture
Explain the evaluation module
Development tools

\subsection{C6678 Cores}
Pipeline
Vector instructions
\subsection{Memory}
Caches
MSMC
Off chip memory in the EVM
\subsection{Multicore Navigator}
QMSS
PDSP
\subsection{Development Tools}


