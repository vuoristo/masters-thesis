\chapter[Texas Instruments TMS320C6678]{Texas Instruments\\TMS320C6678}
\label{chapter:c6678}
This chapter describes the Texas Instruments TMS320C6678 multicore DSP. First
the selection of the TMS320C6678 as the hardware platform for the experiments
is explained. Second the key hardware features of the platform and
the related development tools are examined.

\section{Selection of the Hardware Platform}
The hardware platform used in the experiments in this thesis is the  
Advantech TMDSEVM6678L TMS320C6678 Evaluation Module. The processor in the
evaluation module is the Texas Instruments TMS320C6678. The TMS320C6678
is a fixed and floating point digital signal processor based on the
Texas Instruments Keystone I architecture. The processor has eight
C66x DSP cores at 1.0 GHz clock frequency \cite{tmsdatasheet}.

The TMS320C6678 was chosen as the hardware platform for the experiments
for two main reasons. First the Keystone I architecture is designed with
stream processing applications in mind \cite{multicorevideo}. Second the
objective of this thesis is to understand hardware and software features
supported by the TMS320C6678.

The specific features in the TMS320C6678 we are interested in are the
support for OpenEM included in the MCSDK for Keystone I devices
\cite{MCSDKbrochure} and the hardware accelerated scheduling and
communication utilized by the OpenEM runtime. 

An additional benefit of choosing TMS320C6678 as the hardware platform for
the experiments is the support for the platform in PREESM
\cite{pelcat2014preesm}.
\section{TMS320C6678 Overview}

\begin{figure}[h!]
    \label{arch_overview}
    \begin{center}
        \includegraphics[width=0.99\textwidth]{images/fbd_SPRS691e.png} 
        \caption{Overview of the TMS320C6678 architecture.}
    \end{center}
\end{figure}

The Keystone I architecture the TMS320C6678 is based on specifies a set of
hardware elements which enable integration of DSP cores, application specific
co-processors and IO \cite{tmsdatasheet}. In the Keystone I architecture there
are multiple ways for the C66x cores to communicate with each other, the memory
and the peripherals. The methods of communication of specific interest for the
experiments in this thesis are communication through shared memory discussed in
\ref{subsec:c66memory} and communication through packet based communication
manager Multicore Navigator introduced in \ref{subsec:multicorenav}.

The development board used for development of the experiment applications and 
measurements was an Advantech TMDXEVM6678L. TMDXEVM6678L is an evaluation module
and hardware reference design platform for the TMS320C6678 multicore DSP. The
evaluation module provides 512 megabytes of DDR3 memory and an onboard XDS100
emulator with USB connectivity along with other features. \cite{evmref} The 
connectivity provided by the onboard emulator simplified developing and
debugging for the platform. The memory on the evaluation module helped us omit
the IO from the application design.

The IDE used for development of the experiment applications is called Code
Composer Studio. CCS version 5.2. is distributed with the evaluation module.
CCS is discussed in detail in \ref{subsec:devtools}.

\subsection{c66x DSP}
TMS320C6678 is a multicore fixed and floating-point DSP. The device supports
core speed up to 1.4 GHz but was set up to run at 1.0 GHz for all the
experiments. The device consists of eight c66x DSPs, on-chip memory and
peripherals. The c66x is based on TMS320C66x ISA which is a VLIW architecture
with 8 functional units. The c66x CPU has 64 general-purpose 32 bit registers.
\cite{sprugh7}

The CPU is capable of dispatching up to eight parallel instructions every
cycle. The instructions dispatched in parallel move through pipeline stages
simultaneously. Pipelining helps eliminate CPU stalls while waiting for memory
operations or other CPU instructions taking multiple cycles complete.
\cite{sprugh7}

\subsection{Memory}
\label{subsec:c66memory}
Each c66x CPU has 32 KB level 1 program cache (L1P), 32 KB level 1 data cache
(L1D) and 512 KB of level 2 cache. Initially after bootup both L1P and L1D are
configured as cache but they can be reconfigured as addressable memory by
software. The L2 memory is always configured as addressable memory after reset
but can be configured as cache by software. \cite{tmsdatasheet} L2 SRAM
addresses are always cached with L1P and L1D whereas external memory addresses
are configured noncacheable by default \cite{cacheguide}.

In PC hardware cache coherence is usually handled automatically by the hardware.
In c66x, however, that is not the case. Each c66x core maintains cache coherence
between its L1 caches and the L2 cache automatically but programmer needs to
manage coherence in most other cases. For example if caching is enabled
for an external memory region shared by two cores, explicit cache coherence
operations need to be performed before each core can read from or write to the
shared region \cite{cacheguide}.

Caches
MSMC
Off chip memory in the EVM
Communication
\subsection{Multicore Navigator}
\label{subsec:multicorenav}
QMSS
PDSP
\subsection{Development Tools}
\label{subsec:devtools}

