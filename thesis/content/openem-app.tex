The OpenEM implementation of the filter application was designed using the PREESM filter application described in \ref{sec:preesmapp} as the starting point. Specifically the OpenEM application has to process the frames in similar manner so that only the scheduling policies between the two programming models should differ. The high-level structure of the OpenEM filter application is presented in the figure \ref{fig:openem_flow}. The application consists of two atomic queues, a parallel queue and three execution objects. Each queue is connected to only one execution object. The execution runs in cycles. A new cycle is started every time a cycle finishes. There are multiple cycles running in parallel, but since the read and merge EOs are atomic, only filter EO is running on multiple cores in parallel.

The design principles introduced in the beginning of this chapter apply to the OpenEM application as well. The OpenEM application was not designed for maximum performance but to demonstrate the suitability of OpenEM as a platform for stream processing applications. The performance of the application could be improved in many ways, for example by minimizing the amount of redundant copying of the frame slices and introducing parallelism to frame reading and merging.  

\begin{figure}[h!]
    \begin{center}
        \usetikzlibrary{chains}

\begin{tikzpicture}[>=latex,node distance=0pt, scale=0.7, every node/.style={transform shape}]
    % the rectangular shape with vertical lines
    \node[rectangle split, rectangle split parts=6,
    draw, rectangle split horizontal,text height=0.5cm,text depth=0.5cm,inner sep=0pt, text width=0.0cm] (atomic1) {};
    \fill[white] ([xshift=-\pgflinewidth,yshift=-\pgflinewidth]atomic1.north west) rectangle ([xshift=-5pt,yshift=\pgflinewidth]atomic1.south);

    \node[draw,ellipse,right= 0pt of atomic1,minimum height=1.0cm,inner xsep=5pt] (readEO) {Read EO};

    \node at (5, 0)[rectangle split, rectangle split parts=6,
    draw, rectangle split horizontal,text height=0.5cm,text depth=0.5cm,inner sep=0pt, text width=0.0cm ] (parallel) {};
    \fill[white] ([xshift=-\pgflinewidth,yshift=-\pgflinewidth]parallel.north west) rectangle ([xshift=-5pt,yshift=\pgflinewidth]parallel.south);

    \node[draw,ellipse,right= 0pt of parallel,minimum height=1.0cm,inner xsep=5pt] (filterEO) {Filter EO};

    \node at (10, 0)[rectangle split, rectangle split parts=6,
    draw, rectangle split horizontal,text height=0.5cm,text depth=0.5cm,inner sep=0pt, text width=0.0cm] (atomic2) {};
    \fill[white] ([xshift=-\pgflinewidth,yshift=-\pgflinewidth]atomic2.north west) rectangle ([xshift=-5pt,yshift=\pgflinewidth]atomic2.south);

    \node[draw,ellipse,right= 0pt of atomic2,minimum height=1.0cm,inner xsep=5pt] (mergeEO) {Merge EO};

    % the arrows and labels
    \draw [->](readEO) to (parallel);
    \draw [->](filterEO) to (atomic2);
    \draw [->](mergeEO) .. controls (16,0) and (15, 2) .. (6, 2) .. controls (-2, 2) and (-3, 0) .. (atomic1);

    \node[align=center,below] at (atomic1.south) {Atomic Queue};
    \node[align=center,below] at (parallel.south) {Parallel Queue};
    \node[align=center,below] at (atomic2.south) {Atomic Queue};
\end{tikzpicture}

        \caption{The OpenEM filter application.}
        \label{fig:openem_flow}
    \end{center}
\end{figure}

The application execution starts with initializing the OpenEM framework and the application data structures. The OpenEM initialization is explained in \ref{subsec:ti_init_layer}. After the queues and the execution objects have been initialized the application creates a number of initial events that are sent to the the read queue. Once the events have been queued the application synchronizes all cores and the actual execution starts.

The Read EO combines the functionality of the read and split actors of the PREESM filter application presented in \ref{fig:preesm_actors}. A frame is read from the input video memory and split in to a number of slices. The functions called are exactly the same as in the PREESM application. The frame slices are copied to event buffers of new events. The new events are then sent to the filter queue. There is only one filter queue and one filter execution object. The same execution object will compute either sobel or gauss filter on the frame slice according to its type.

The PREESM application splits the frames into slices and processes the slices separately before merging them back into one frame. Similar fork-join mechanism was implemented in the OpenEM application. Event groups were first planned to be used as the fork-join mechanism in the filter application but in the final implementation a different, simpler mechanism was used. The TI implementation of event groups lacks \texttt{em\_event\_group\_delete} function which makes it necessary to reuse the existing event groups. The example applications which are included in the NSN OpenEM distribution described in \ref{sec:emframework} demonstrates the reuse of event groups, but it was estimated that the programming overhead resulting from the reuse of the groups would be larger than implementing the fork-join in a simpler manner. In the final implementation the frames are accumulated simply in a merge buffer located in shared memory which is referenced through queue context pointers. The bookkeeping for frame completion is handled in the same location.

The filtered frame slices are accumulated in the merge buffer, which is accessed through the merge queue context. The merge queue is atomic to avoid need for synchronization between the cores. After a frame has been merged the merge EO creates a new event which is sent to the read queue. The ratio of sobel filtered frames to gauss filtered frames is always one to one.
