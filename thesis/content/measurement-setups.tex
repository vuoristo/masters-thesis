The filter applications process two video streams simultaneously as described in chapter \ref{chapter:construction}. One video stream is processed with a sobel filter and the other is processed with a gaussian filter. The dynamic behavior of the applications is investigated using different workloads.

\begin{table}
    \begin{center}
        \begin{tabular}{ c c }
            Sobel Resolution & Gauss Resolution \\ \hline
            CIF              & CIF              \\ \hline
            4CIF             & CIF              \\ \hline
            CIF              & 4CIF             \\ \hline
            QCIF             & QCIF             \\ \hline
        \end{tabular}
        \caption{PREESM and OpenEM measurement setups}
        \label{tab:preesm_setups}
    \end{center}
\end{table}

\begin{itemize}
    \item \textit{Comparing Dynamic and Static Scheduling}
    The workloads used in the first experiment are presented in the table \ref{tab:preesm_setups}. The purpose of the different bitrates used for each video stream is to expose the behavior of the OpenEM scheduler in handling dynamic workloads. The static schedule in the PREESM application will provide a baseline to reflect the OpenEM performance to, but the performance of the applications should not be directly compared due to the difference in the runtime systems.

\item \textit{Investigating the Balance of Latency and Throughput}
To examine the capabilities of the dynamic scheduler the number of simultaneously processed events is varied in the second experiment. The maximum number of frames that can be processed simultaneously corresponds to the number of initial events created in the initialization. Both filters are loaded with CIF streams and the measurements are run with 1 to 24 initial events. The number of simultaneous frames changes the throughput versus latency balance of the OpenEM application.

\item \textit{Examining the Efficiency of Parallel Scheduling}
In the third experiment the OpenEM application is measured with different core masks to investigate the ability of the dynamic scheduler to utilize the increased parallel resources. Both filters of the OpenEM application are loaded with CIF streams. Different numbers of cores are used, but all actors can run on all available cores. The experiment is run with core masks allowing one to eight cores be used for processing the streams. The experiment is run with 16 initial events.
\end{itemize}
