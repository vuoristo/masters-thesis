Analysing the performance of OpenEM based stream processing applications is a broad task, requiring the use of many distinct technologies. First a suitable workload, which is representative of stream processing tasks had to be designed. Next two versions of the workload application had to be implemented, one for the evaluation of OpenEM and the other to provide a point of comparison. In order to get usable data from the applications, they had to be instrumented and the measurement data analysed. Most of the work required was straightforward but some complications were faced with the tools.

The baseline application was implemented using PREESM. PREESM generates static schedules for the application from provided the dataflow graph and the source code. To get specific points of comparison for each of the measurement setups working with the PREESM automatic code generation was complicated. The tool would create different schedules on each iteration and thus yield applications with dissimilar core usage between the measurement setups.

Texas Instruments OpenEM version 1.0.0.2 that was used for the implementation of the OpenEM workload has some unimplemented features. Most of the unimplemented features were documented and did not cause unnecessary work. However the implementation of Queue Groups did not work as expected. Creation of multiple queue groups did not work as specified in the documentation, and was left unexplored after multiple trials at working with them.

The documentation of the Texas Instruments implementation of OpenEM was lacking. The documentation provided with the source code was incomplete and included a sketchy version of the OpenEM white paper~\cite{moerman2014open} which was the most important source for understanding the use of hardware acceleration in the runtime. The documentation also lacked details for the initialization layer without which the initialization of the hardware components would have been difficult. The initialization code was provided as a part of a scarcely documented example application.
