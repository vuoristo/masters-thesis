The definition of the Dataflow term is bit fuzzy but the common idea between concepts under the term is to divide computation into nodes that can be executed concurrently. There are dataflow programs, hardware architectures and other things labeled under dataflow. The Dynamic Dataflow concept studied here falls under the dataflow models of computation.

A program built using the Dataflow MoC consists of a directed graph with data flowing between the actors. The data is split into tokens that are passed between the actors. Since the execution of the actors is asynchronous, the tokens have to be buffered between the actors. The dataflow MoC allows for unbounded execution of the model, which means the dataflow program may be able to execute for a very long time. The possibility of unbounded execution leads to the problems the different dataflow MoCs try to solve. A dataflow program capable of unbounded execution 1. may have unbounded buffer growth 2. if there are cycles, the execution may result in a deadlock where there are not enough tokens to advance the execution.

One popular approach to solving these problems is the Synchronous Dataflow (SDF). In SDF the number of tokens produced and consumed by each actor is fixed.  The SDF MoC guarantees bounded buffers and deadlock-free execution but it is very constrained. For expressing more complicated programs, models with more design freedom are needed.

There exists a variety of Dataflow MoCs that extend the concept of Synchronous Dataflow such as PiSDF, which extends SDF expressive power by defining parameters and interfaces. The resulting PiSDF model can be expressed as a SDF. We will not look at these extensions of SDF but at the more generic Dataflow MoCs categorized under Dynamic Dataflow. Dynamic Dataflow does not refer to a single MoC but is rather an umbrella term under which many MoCs fall.

In Dynamic Dataflow (DDF) the number of tokens consumed and produced by an actor in a single firing is not constrained. An actor can produce and consume different number of tokens on every firing. This freedom improves the expression power of the model but makes the analysis more difficult. Buck (1993) proves in his thesis that bounded buffers and deadlocks are not decidable for DDFs.

In the Handbook of Signal Processing Systems Bhattacharyya et al. describe many examples of DDF MoCs. One of these examples is the CAL Actor Language (CAL). CAL is used for example by the MPEG Reconfigurable Video Coding library.

Another example use of Dynamic Dataflow is in the TensorFlow (TF) machine learning library by Google. The TF programs are constructed around a DDF graph.  The computations in the nodes can be distributed to heterogeneous computing devices such as CPUs and GPUs. TF provides control flow operators that can be added to the graph to support conditional execution of parts of the graph and loops.

Introduction to embedded \cite{lee2015introduction}

Buck - PHD on dynamic dataflow http://ptolemy.eecs.berkeley.edu/publications/papers/93/jbuckThesis/thesis.pdf
\cite{buck1993scheduling}

PiSDF https://halshs.archives-ouvertes.fr/hal-01075114/document
\cite{desnos2013pimm}

Handbook of Signale Processing Systems HSPS12
\cite{bhattacharyya2013handbook}

CAL \cite{eker2003cal}

tf \cite{tensorflow2015-whitepaper}

\fixme{todos}
Define dataflow
Define dataflow moc
Motivate dataflow in context of stream computation
Define synchronous dataflow
Give examples of SDF
Define Dynamic dataflow
Give examples of DDF

\fixme{end of todos}
