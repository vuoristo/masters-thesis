\fixme{set the dataflow mocs in larger context}

Stream processing defines the high level structure of how streaming data is to be processed but it does not define the stream processing methods in detail. In this section Dataflow Models of Computation are examined as an example of stream processing paradigm.

\subsection{Dataflow Models of Computation}
\label{subsec:dataflow-moc}
Dataflow models of computation can be used to describe stream processing applications. In programs following a dataflow MoC, the computation can be described by a directed graph. The nodes of the graph are the computation kernels. The data being processed by the application is split into tokens which flow from node to node along the directed arcs. The only means of communication between the nodes are the data tokens. This means a node may begin executing as soon as it has received its input tokens. The execution of the nodes is thus asynchronous.~\cite{lee2015introduction}

To allow asynchronous execution of the nodes the tokens are buffered between the nodes. A common requirement of MoCs for used in signal and stream processing is the capability of executing for an undetermined amount of time. Execution without determinate end is called unbounded execution and it means that the length of the input may be arbitrarily large. The token buffering combined with the required capability for unbounded execution leads to a possibility of unbounded buffer growth. Unbounded buffer growth is one of the main problems the different dataflow MoCs try to solve. The other main problem is the schedulability of the dataflow graph. If the graph has cycles, the model may deadlock when one of the nodes has an insufficient number of input tokens and cannot proceed execution. This keeps the nodes that come after it from receiving input and thus deadlocking the graph.~\cite{lee2015introduction}

The scheduling of dataflow graphs means determining when and where the nodes execute. The execution of the nodes is dependent on the availability of data, which places restrictions on the possible schedules. The availability of data means that the node must have the specified number of tokens in each of its input queues in order to execute. Depending on the system there may be one or more cores available for execution of the nodes. For parallel systems the nodes may execute in parallel if a parallel schedule can be found that respects the data availability constraints of the graph. Communication between cores may introduce overhead that the scheduler must be aware of in order to generate efficient schedules.~\cite{lee1987synchronous}

\subsection{Synchronous Dataflow}
\label{subsec:synchronous-dataflow}
\FloatBarrier
One approach to solving the schedulability and the unbounded buffer growth of the dataflow MoC is fixing the production and consumption rates of the actors. These limitations are implemented in the dataflow models introduced by Lee and Messerschmit called synchronous dataflow (SDF)~\cite{lee1987synchronous}. The SDF model imposes strict limits on the kinds of dataflow graphs allowed. The dataflow graphs in SDF models are fixed at compile time. No new actors may be added in run time to SDF graphs because all arcs of the model are fixed. The flow of tokens along the arcs is fixed as well, due to the fixed production and consumption rates of the model. Creating graphs adhering to these limitations, the graphs are fully schedulable at compile time and they execute the same way on each iteration. This means that the boundedness of the buffers is guaranteed and the deadlocking problems are avoided. Implementing such limitations reduces the expression power of SDF models compared to more lax dataflow models such as the dynamic dataflow discussed in~\ref{subsec:dynamic-dataflow}.~\cite{lee2015introduction}

The SDF MoC was specifically designed for digital signal processing applications. In domains such as digital signal processing the typical tasks can be presented as sequence of operations performed on an infinite stream of data, which makes them computable with SDF models. The SDF models provide good performance and guarantee deadlock-free execution and are thus often a practical choice for DSP applications.~\cite{lee2015introduction}

\begin{figure}[h!]
    \begin{center}
        \includegraphics[width=0.55\textwidth]{images/sdf_graph.pdf}
        \caption{A SDF Graph from \cite{ade1997data}.}
        \label{fig:sdf_graph}
    \end{center}
\end{figure}

\begin{table}
    \begin{center}
        \begin{tabular}{ c c }
            Arc & Buffer Size \\ \hline
            ab  & 5 \\ \hline
            ac  & 10 \\ \hline
            bd  & 2 \\ \hline
            cd  & 10 \\ \hline
            ce  & 12 \\ \hline
            cf  & 30 \\ \hline
            df  & 6 \\ \hline
            ef  & 8 \\ \hline
        \end{tabular}
        \caption{The minimum required buffer sizes for the arcs in the SDF graph \ref{fig:sdf_graph}. The buffer sizes are presented as the number of tokens the buffer fits at maximum.}
        \label{tab:sdf_buffers}
    \end{center}
\end{table}

Figure~\ref{fig:sdf_graph} presents an example acyclic SDF graph from~\cite{ade1997data}. The graph nodes are marked with letters from a to f and the production and consumption rates are marked at the ends of the arcs. The graph can be scheduled in many different ways, which yield different performance characteristics in terms of memory requirements of the buffers, latency and throughput. The task of scheduling and allocating buffers for SDF graphs is non-trivial. Larger buffer sizes allow for more scheduling freedom but on certain targets such as FPGAs there are severe limitations on the number of registers available for the buffer allocation. The minimum required buffer sizes that guarantee the existence of deadlock-free schedule are represented in table~\ref{tab:sdf_buffers}. The buffer sizes were found using the algorithm from~\cite{ade1997data}. For some schedules that would yield improvements in throughput or latency, larger buffer sizes may be required.

Examples of practical implementations of the SDF MoC include the parameterized and interfaced synchronous dataflow (PiSDF) used by PREESM described in~\ref{subsec:preesm-internal}, the Ptolemy project \cite{ekerjanneck2003ptolemy} and the LUSTRE programming language~\cite{halbwachs1991synchronous}.

\FloatBarrier
\subsection{Dynamic Dataflow}
\label{subsec:dynamic-dataflow}
Synchronous dataflow may be a good fit for simple signal processing tasks such as signal filtering, but it is too restricted for efficiently describing more complicated algorithms. Multiple models of computation that share the basic structure of dataflow models have been developed for more advanced needs. Dataflow models of computation that relax the constraints on token production and consumption are grouped under the name dynamic dataflow. Dynamic dataflow does not refer to a single model of computation but a group of models that differ in terms of expression power and analysability.~\cite{bhattacharyya2013handbook}

The dynamic dataflow models of computation do not constrain the number of tokens consumed or produced by an actor in a single firing. The actors may produce and consume different numbers of tokens on different firings. Relaxation of these constraints improves the expression power of the model but makes the analysis more difficult. In the class of dataflow models where these constraints have been lifted the boundedness of the buffers and the deadlocks of the cyclic graphs are undecidable~\cite{buck1993scheduling}. Many dynamic dataflow models introduce other kinds of restrictions to replace the constraints on production of tokens. By limiting the types of the actors and available graph patterns dynamic dataflow models with decidable buffer growth and schedules are possible \cite{bhattacharyya2013handbook, gao1992well}. Choosing which dynamic dataflow model to use is about finding the right tradeoff between analysability and expression power for the specific application.

Dynamic dataflow is suitable for processing streams of complex data, such as compressed audio or video. A common example use case for DDF in the literature is the decoder of MPEG video stream~\cite{bhattacharyya2013handbook}. CAL actor language supports DDF MoC and it is used by the reconfigurable video coding working group to describe the MPEG encoder and decoder~\cite{bhattacharyya2011overview}. A more recent example of using DDF in practice is in the TensorFlow machine learning framework developed by Google~\cite{tensorflow2015-whitepaper}.
