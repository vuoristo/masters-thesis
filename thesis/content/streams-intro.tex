\fixme{todo \\
    what does dataflow have to do with anything
    \url{http://venturebeat.com/2014/03/19/without-stream-processing-theres-no-big-data-and-no-internet-of-things/} \\
    \url{https://www.quora.com/Whats-the-difference-between-real-time-processing-and-stream-processing} \\
    \url{http://www.emc.com/leadership/digital-universe/2014iview/executive-summary.htm}
}

The amount of digital data that is being created and copied is increasing at a massive pace. EMC Digital Universe study \cite{turner2014digital} estimates that the data we create and copy annually was 4.4 zettabytes in 2013 and is going to reach 44 zettabytes by 2020. A large portion of this data growth is due to increased volume of entertainment in the form of audio and video being streamed through internet. New data sources such as embedded sensors are contributing to the explosive data growth as well. Most of the data being created or copied is transient and thus does not require long term storage.~\cite{turner2014digital}

Distributed data processing systems such as Hadoop~\cite{white2012hadoop} were developed for batch processing of Big Data. The batch processing systems are capable of processing massive amounts of data but the processing has high latency. As most of the growth in data creation and copying is coming from data that is not stored, batch processing is not optimal processing method of the data.

Consider live streaming of video as an example of modern application with specific data processing needs. The data is being generated by a camera filming a live scene and the users expect to access the video stream over the internet with low latency. Many distinct processing steps are required in order to get the video frames from the camera to the viewer, all of which need to be performed in few milliseconds on the frames flowing past the processing unit. In addition to video streams many other kinds of data streams which have similar processing needs are produced at an accelerating pace. Certain technologies developed for processing the streaming data are grouped under the term stream processing.

An introduction to stream processing and selected stream processing techniques are given in this chapter. A stream processing overview is given in \ref{subsec:stream-processing-overview}. Selected streaming applications and their common features are described in \ref{subsec:streaming-applications}. Video streams are discussed as an example of a common type of streaming data in \ref{subsec:video-streams}. Dataflow Models of Computation are discussed in \ref{subsec:dataflow-moc} as an example of solution for stream processing. Variations of dataflow MoC are examined, synchronous dataflow in \ref{subsec:synchronous-dataflow} and dynamic dataflow in \ref{subsec:dynamic-dataflow}.
