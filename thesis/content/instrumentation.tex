To understand the behavior and performance of the applications their execution needs to be measured. The measurements conducted in the experiments were based on cycle counts. The TMS320C6678 exposes cycle counts on each core through two registers, namely TSCL and TSCH which store the low 32 bits and the high 32 bits of the cycle count respectively. Comparing the values of the TSCL register in the beginning and the end of processing on each core gives accurate local timings and would be enough for measuring the execution on single core. However for accurate measurements of global cycles, the cycle offsets to the global cycle count needed to be stored. The functionality for counting cycles was included in the Texas Instruments OpenEM initialization and utilities code. The cycle counts were stored in the DDR memory.

In the PREESM application the cycle counts were saved in the beginning and the end of each user defined function of the application. In addition to this total cycle count was measured in the \texttt{readYUV} function.  

In the OpenEM application cycle counts were measured over the same functions as in the PREESM application and in the beginning and in the end of each EO. The latter measurements were carried out to get a better understanding of the OpenEM runtime overhead.

The measurements were exported through the debug connection of the TMS320C6678 using I/O library provided by Texas Instruments.
