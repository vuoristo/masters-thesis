Many different video stream formats are needed to capture, transfer and store videos. Raw video material requires a lot of space to store and a lot of bandwidth to transfer but when the video data is edited, the part of the video stream that is being edited has to be in unpacked format. In this section two standards related to video streams are explained. First the YUV color specification and second the CIF frame resolution format.

\subsection{YUV Format}
\label{subsec:yuv}
YUV color space is used in encoding colors of images and videos. The YUV color system is used for example in encoding the colors of television broadcasts using the PAL or NTSC systems. The YUV color space is designed with the human perception in mind, meaning the channels are selected so that compression artifacts and other errors are more likely masked by human perception. This allows for reduced bandwith compared to RGB encoded colors. The Y channel corresponds to the luminance of the image. Luminance means the perceived brightness. In a black and white image or video only the Y channel is used. The U and V channels are called the chrominance channels and they encode the color component of the image.~\cite{jack2011video}

In digital media YUV term is commonly used to refer to YCbCr, which is a way of encoding RGB color information. Y' (Y prime) is called luma and is distinct from the Y channel of the analog YUV system. The luma channel is a non-linear encoding of the light intensity. Cb and Cr are the blue-difference and red-difference components respectively. It is common to store the luma channel at a higher resolution than the chroma channels to save bandwidth. This process is called chroma sub-sampling. Humans are more sensitive to the brightness of the image than the color of the image and thus the lower resolution of the chroma channels causes less noticable artifacts.~\cite{jack2011video}

\subsection{Common Intermediate Format}
\label{subsec:cif}
Common Intermediate Format or CIF is a standardization of the horizontal and vertical resolutions of pixels in video signals. The CIF pixels are non-square with an aspect ratio of approximately 1.222:1. The standard defines multiple resolutions such as the CIF 352x288 and QCIF for quarter CIF 176x144.CIF resolutions were designed to be easily convertible to the PAL and NTSC systems.~\cite{telecommunication1993itu}
