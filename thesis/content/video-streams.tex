\subsection{YUV Format}
\label{subsec:yuv}
YUV color space is used in encoding colors of images and videos. The YUV color system is used for example in encoding the colors of television broadcasts using the PAL or NTSC systems. The YUV color space is designed with the human perception in mind, meaning the channels are selected so that compression artifacts and other errors are more likely masked by human perception. This allows for reduced bandwith compared to RGB encoded colors. The Y channel corresponds to the luminance of the image. Luminance means the perceived brightness. In a black and white image or video only the Y channel is used. The U and V channels are called the chrominance channels and they encode the color component of the image.~\cite{jack2011video}

\subsection{Common Intermediate Format}
\label{subsec:cif}
Common Intermediate Format or CIF is a standardization of the horizontal and vertical resolutions of pixels in video signals. The CIF pixels are non-square with an aspect ratio of approximately 1.222:1. The standard defines multiple resolutions such as the CIF 352x288 and QCIF for quarter CIF 176x144.~\cite{telecommunication1993itu}
