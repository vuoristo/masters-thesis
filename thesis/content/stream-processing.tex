To answer to the challenge set by the increasing amount of streaming data, stream processing is needed. Stream processing term has been used in the literature to describe different kinds of methods that deal with streaming data. In the context of this thesis the broad definition of stream processing as a method of processing any kind of streaming data is used. A survey of the other meanings of stream processing is given in~\cite{stephens1997survey}.

\subsection{Stream Processing Systems}
\label{subsec:stream-processing-systems}
The systems built following stream processing programming paradigm can be categorized under stream processing systems. Stream processing systems consist of modules that compute in parallel and communicate data via channels. The modules of stream processing systems can be divided into three classes according to their placement and purpose in the system. The classes are sources, filters and sinks. Sources act as the input points that pass data into the system. Filters perform atomic computations on the streams. Sinks are used to pass the data out from the system.~\cite{stephens1997survey}

The channels of communication between the modules of stream processing systems are called streams. Streams can be described as infinite lists of elements taken from a dataset $A$. Mathematical formalization of a stream is a function $f:T \rightarrow A$ where $T = \mathbb{N}$ represents discrete time.~\cite{stephens1997survey} For example the input stream of a signal processing application can be the sample based input sequence which has been generated by sampling a sensor at fixed time intervals.

The stream processing survey by Stephens \cite{stephens1997survey} categorizes \textit{dataflow systems}, \textit{reactive systems}, \textit{synchronous concurrent algorithms}, \textit{signal processing systems}, and some \textit{real-time systems} as stream processing systems. A closer look at dataflow models is taken in~\ref{sec:dataflow-models}.

In addition to the software stream processing systems studied in this thesis, the increasing volume of streaming data has motivated research for hardware stream processors. Processor architectures that implement stream processing concepts in the hardware have been researched in Imagine \cite{kapasi2002imagine} and Merrimac~\cite{dally2003merrimac} projects at Stanford University.

\subsection{Streaming Applications}
\label{subsec:streaming-applications}
Stream processing is not limited to a specific application domain such as signal processing applications. Diverse variety of stream processing applications may run on different platforms ranging from phones to servers. The authors of the StreamIt language \cite{thies2002streamit} have defined \textit{streaming applications} as a class of programs, that commonly have many of the features defined in the following listing.

\begin{itemize}
    \item \textit{Large streams of data.} A streaming application operates on large, virtually infinite streams of data.
    \item \textit{Independent stream filters.} The filters of a streaming application are generally self-contained. They perform atomic operations on the stream.
    \item \textit{A stable computation pattern.} A streaming application has a steady state of operation during which the graph formed by the filters remains mostly constant.
    \item \textit{Occasional modification of the stream structure.} A streaming application can occasionally modify the processing graph as a reaction to changed input or some other condition.
    \item \textit{Occasional out-of-stream communication.} The high volume communication between the filters is handled through the streams but the filters may communicate small amount of control data outside the stream.
    \item \textit{High performance expectations.} There often are real-time and power consumption constraints on streaming applications. For example a streaming video decoder has to decode the stream at rate of input in order to avoid unbounded buffer growth or frame dropping.
\end{itemize}

Stream processing frameworks have been developed for distributed computing. Examples of stream processing frameworks are Google MillWheel~\cite{tyler2013millwheel}, Apache Storm~\cite{apache2016storm} and Apache Spark Streaming~\cite{apache2016spark}. GPU hardware is suitable for stream processing, for example~\cite{goddeke2011fast} makes use of GPUs for solving simulations involving partial differential equations using GPUs.

\subsection{Video Streams}
\label{subsec:video-streams}
Video streams are a prime example of streaming data. In the case of streaming service such as Netflix, the video data is downloaded from the service providers server and decompressed on the device of the consumer. In most use cases the video streams are decompressed as they are downloaded and the complete video file is never stored on the device. Thus stream processing is well suited for video stream decompression. The video streams are often accompanied by audio streams, which are processed similarly.~\cite{richardson2002video}

The video conference use case is similar to the streaming video services but involves extra complexity with the compression of the raw data coming from the video camera and especially the requirement of low latency in the compression, decompression and communications.

A large fraction of the growth of data creation can be attributed to video streams. More than billion hours of TV and movies are streamed through the video streaming service Netflix \cite{turner2014digital}. Video streams are used for real-time communication through services such as Skype and Periscope. The efficiency of video stream processing is thus a top priority in the industry.
