Stream processing is a method of computing over streaming data. Stream processing is often used to refer to the programming paradigm that follows the stream processing method but stream processing is not limited to the creation of software. Stream processing term has been used in the literature to describe different kinds of methods that deal with streaming data. In the context of this thesis the broad definition of stream processing as a method of processing any kind of streaming data is used. 

Stream processing paradigm defines modules that compute in parallel and communicate data via channels. The modules can be divided into three classes according to their placement and purpose in the process. The classes are sources, filters and sinks. Sources act as the input points that pass data into the process. Filters perform atomic computations on the streams. Sinks are used to pass the data out from the process.~\cite{stephens1997survey}

In stream processing the channels of communication between the modules are called streams. Streams can be described as infinite lists of elements taken from a dataset $A$. Mathematical formalization of a stream is a function $f:T \rightarrow A$ where $T = \mathbb{N}$ represents discrete time.~\cite{stephens1997survey} For example the input stream of a signal processing application can be the sample based input sequence which has been generated by sampling a sensor at fixed time intervals.

The systems built following the stream processing method can be categorized under stream processing systems. The stream processing survey by Stephens \cite{stephens1997survey} categorizes \textit{dataflow systems}, \textit{reactive systems}, \textit{synchronous concurrent algorithms}, \textit{signal processing systems}, and some \textit{real-time systems} as stream processing systems. 

\subsection{Applications of Stream Processing}
\label{subsec:streaming-applications}
The authors of the StreamIt language~\cite{thies2002streamit} have defined \textit{streaming applications} as a class of programs, that commonly have many of the features defined in the following listing.

\begin{itemize}
    \item \textit{Large streams of data.} A streaming application operates on large, virtually infinite streams of data.
    \item \textit{Independent stream filters.} The filters of a streaming application are generally self-contained. They perform atomic operations on the stream.
    \item \textit{A stable computation pattern.} A streaming application has a steady state of operation during which the graph formed by the filters remains mostly constant.
    \item \textit{Occasional modification of the stream structure.} A streaming application can occasionally modify the processing graph as a reaction to changed input or some other condition.
    \item \textit{Occasional out-of-stream communication.} The high volume communication between the filters is handled through the streams but the filters may communicate small amount of control data outside the stream.
    \item \textit{High performance expectations.} There often are real-time and power consumption constraints on streaming applications. For example a streaming video decoder has to decode the stream at rate of input in order to avoid unbounded buffer growth or frame dropping.
\end{itemize}

Applications that handle streams of audio and video can often be implemented using stream processing paradigm. The multimedia domain is therefore full of examples of stream processing applications. Video conferencing is a often used example of a streaming application. Streaming video decoding is used in online video streaming services. Audio streams are encoded and decoded in mobile devices for calls but also in streaming of music.

In addition to multimedia, streaming data is often encountered in other kinds of internet services as well. For example efficient computation of analytics from search engine data can be implemented using stream processing. The total dataset may be quite large and a batch computing system computing over the complete data would provide results with high latency. A stream processing system however would be able to compute the analytics from the data as the data is being produced and update the analytics separately for each piece of data received. Google has developed the MillWheel framework for implementation of such analytics~\cite{tyler2013millwheel}.

\subsection{Stream Processing Platforms}
\label{subsec:stream-processing-platforms}
A diverse variety of stream processing applications may run on different platforms ranging from phones to servers. Software stream processing systems may execute on arbitrary hardware, but to achieve good performance some platforms are preferred over the others. Stream processing is well suited for designing applications for GPUs and DSPs. The modules of stream processing can often be executed in parallel allowing for efficient use of the multiple cores of GPUs, for example~\cite{goddeke2011fast} makes use of GPUs for solving simulations involving partial differential equations using GPUs.

Digital signal processing applications are often designed in stream processing pattern and this is also reflected in the hardware making DSPs potentially powerful stream processing units~\cite{lee2015introduction}. Multicore DSPs such as the Texas instruments TMS320C6678 used in this thesis have the potential for efficient stream processing, because they combine the DSP architecture suitable for stream processing with the parallel processing capabilities of the multiple cores.

In the software world the expression power of stream processing paradigm has been recognized and many tools have been created for the development of stream processing applications. Examples of stream processing frameworks for distributed computing are Google MillWheel~\cite{tyler2013millwheel}, Apache Storm~\cite{apache2016storm} and Apache Spark Streaming~\cite{apache2016spark}. 

In addition to the software stream processing systems studied in this thesis, the increasing volume of streaming data has motivated research for hardware stream processors. Processor architectures that implement stream processing concepts in the hardware have been researched in Imagine \cite{kapasi2002imagine} and Merrimac~\cite{dally2003merrimac} projects at Stanford University.

\subsection{Video Streams}
\label{subsec:video-streams}
Video streams are a prime example of streaming data. In the case of streaming service such as Netflix, the video data is downloaded from the service providers server and decompressed on the device of the consumer. In most use cases the video streams are decompressed as they are downloaded and the complete video file is never stored on the device. Thus stream processing is well suited for video stream decompression. The video streams are often accompanied by audio streams, which are processed similarly.~\cite{richardson2002video}

The video conference use case is similar to the streaming video services but involves extra complexity with the compression of the raw data coming from the video camera and especially the requirement of low latency in the compression, decompression and communications.

A large fraction of the growth of data creation can be attributed to video streams. More than billion hours of TV and movies are streamed through the video streaming service Netflix \cite{turner2014digital}. Video streams are used for real-time communication through services such as Skype and Periscope. In addition to the number of users accessing the streams growing the bitrate of the streams is growing as well. The efficiency of video stream processing is thus a top priority in the industry.
