Edge detection is an important tool in image processing and computer vision. Many image processing and computer vision algorithms operate on detected edges. The quality of the detected edges is important for the algorithms using them as input. In his 1986 paper John F. Canny \cite{canny1986computational} lays out the mathematical criteria for successful edge detection and presents an algorithm, which achieves decent edge detection performance. The algorithm is suitable for implementation on DSPs. The Canny edge detection algorithm consists of five steps presented in the following list.

\begin{enumerate}
    \item{Noise reduction}
    \item{Finding the intensity gradient of the image}
    \item{Non-maximum suppression}
    \item{Double tresholding}
    \item{Edge tracking by hysteresis}
\end{enumerate}

In the canny edge detector the image is first filtered with a gaussian filter to reduce the amount of noise in the image. Second, the changes in the intensity in the image are detected using an image gradient operator such as the sobel operator. The third step improves the accuracy of the edge detection by supressing all but the strongest responses to the detected edges, in practice ``thinning'' the edges. The fourth step classifies the edge pixels to three classes separated by empirically determined threshold values. The pixels with gradient value above the high threshold are marked strong pixels and the pixels with gradient value below the low threshold are supressed. In the fifth step the remaining weak pixels with gradient values below the high threshold are preserved or supressed according to the presence of strong pixels in their neighborhood. Detailed description of the algorithm is presented in the original paper by Canny \cite{canny1986computational}, information about implementing a canny edge detector is available in \cite{gonzalez2008digital} and comparison of its performace to other edge detectors can be found in \cite{maini2009study}.

In this section the phases of canny edge detector are described in the order they are used in Canny filter. First, the gaussian filter is introduced in subsection \ref{subsec:gauss}. The sobel filter is looked at next in subsection \ref{subsec:sobel}. After the sobel filter, the edge responses are pruned in three phases. The three phases are presented in subsection \ref{subsec:canny-edge-pruning}.

\subsection{Gaussian filter}
\label{subsec:gauss}
Gaussian filtering is used for multiple purposes in digital image processing. In the canny edge detector the gaussian filter is used to reduce noise in the processed images. The gaussian filter works by convolving a gaussian function with the input signal. Gaussian function is non-zero everywhere which means it would theoretically require an infinite convolution window. Since the function decays rapidly it is often reasonable to truncate the function and use small windows.~\cite{gonzalez2008digital} 

Calculating the convolution with a truncated function means in practice that every pixel in the filtered image has an intensity value computed by taking a weighted average of the neighboring pixels in the input image. The weights are pre-calculated from the gaussian function, giving the highest weight to the pixel in the center of the window. The gaussian filter displayed in figure \ref{fig:gaussmat} was calculated with $\sigma$ = 1.3. The filtered image has a smoothed appearance compared to the original image.

\begin{figure}
    \begin{displaymath}
        B = \frac{1}{159}\begin{bmatrix}
             2 & 4 & 5 & 4 & 2 \\
             4 & 9 & 12 & 9 & 4 \\
             5 & 12 & 15 & 12 & 5 \\
             4 & 9 & 12 & 9 & 4 \\
             2 & 4 & 5 & 4 & 2 \\
        \end{bmatrix} \ast A
    \end{displaymath}
    \caption{Convolution with a gaussian kernel computed with $\sigma = 1.3$. The convolution operation is denoted by the asterisk.}
    \label{fig:gaussmat}
\end{figure}

\subsection{Sobel filter}
\label{subsec:sobel}
The actual edge detection in the canny edge detector begins with determining the changes in the intensity of the image. This is done by applying the sobel operator to the input image. The sobel operator is a discrete differentiation operator. It consists of two 3x3 kernels which are convolved with the image to approximate the derivatives. The two kernels represent horizontal and vertical changes. At each point in the image the resulting gradient approximations are combined giving an approximate gradient magnitude. \cite{gonzalez2008digital} The convolution operations are presented in figure \ref{fig:sobelmat}.

\begin{figure}
    \begin{displaymath}
        G_{x} = \begin{bmatrix}
            -1 & 0 & +1 \\
            -2 & 0 & +2 \\
            -1 & 0 & +1 \\
        \end{bmatrix} \ast A
    \end{displaymath}
    \begin{displaymath}
        G_{y} = \begin{bmatrix}
            -1 & -2 & -1 \\
            0 & 0 & 0 \\
            +1 & +2 & +1 \\
        \end{bmatrix} \ast A
    \end{displaymath}
    \caption{The 3x3 Sobel kernels used in the application to compute the gradient approximation. $G_{x}$ is the horizontal gradient approximation at given pixel and $G_{y}$ is the vertical gradient approximation. The asterisk denotes the convolution operation.}
    \label{fig:sobelmat}
\end{figure}

\subsection{Canny Edge Pruning}
\label{subsec:canny-edge-pruning}
Canny edge detector is designed to detect edges accurately and as unambiguously as possible. To make unambiguous detections, each edge in the input image should produce only one edge response. For this purpose Canny edge detector employs \textbf{non-maximum suppression}. Non-maximum suppression works on the gradient image that was calculated in the previous phase. For each pixel in the gradient images, non-maximum suppression compares the gradient value to the adjacent pixels in positive and negative gradient directions. The pixel is preserved only if it has the largest gradient value compared to its neighbors.~\cite{gonzalez2008digital}

The non-maximum suppression outputs an image where the edge responses have been pruned to single response per edge, but there still remain gradient pixels that correspond to noise or other uninteresting variations in the image. The detection quality is further improved by applying two thresholds to the output of the non-maximum suppression phase. In the literature this phase is called \textbf{double thresholding}. If the gradient value of a pixel is higher than the high threshold, the pixel is marked as a strong pixel. If the value is lower than the high threshold but higher than the low threshold, the pixel is marked as weak pixel. Pixels with gradient values lower than the low threshold are automatically discarded. The threshold values are determined empirically.~\cite{gonzalez2008digital}

In the final phase of the Canny edge detector, called \textbf{edge tracking by hysteresis} the strong pixels are used to determine which of the weak pixels to keep and which to discard. A weak pixel is preserved if there is at least one strong pixel among its eight neighboring pixels. If there are no strong pixels in the neighborhood the weak pixel is discarded.~\cite{gonzalez2008digital}
