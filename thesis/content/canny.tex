Edge detection is an important tool in image processing and computer vision. Many image processing and computer vision algorithms operate on detected edges. There is a growing interest in the industry to use DSP platforms for edge detection based algorithms. \fixme{find source for this claim. is this only TI marketing?}

In his 1986 paper John F. Canny \cite{canny1986computational} lays out the mathematical criteria for successful edge detection and presents an algorithm, which is suitable for implementation on DSP platforms and achieves decent edge detection performance. The Canny Edge Detection consists of five steps presented in the following list.

\begin{enumerate}
    \item{Noise reduction}
    \item{Finding the intensity gradient of the image}
    \item{Non-maximum supression}
    \item{Double tresholding}
    \item{Edge tracking by hysteresis}
\end{enumerate}

The noise reduction in step one can be implemented with gaussian filtering. Finding the intensity gradient of the image in step two can be implemented using a sobel filter. Gauss and sobel filters are implemented in the thesis experiment. Gaussian filtering is discussed in \ref{subsec:gauss} and sobel filtering in \ref{subsec:sobel}. The rest of the steps are not implemented in this thesis and thus are only briefly discussed here \fixme{subsection for the rest of the canny steps}.

In the canny edge detector the image is first filtered with gaussian filter to reduce the amount of noise in the image. Second, the changes in the intensity in the image are detected using an edge detection operator such as the sobel operator. The third step improves the accuracy of the edge detection by supressing all but the strongest responses to the detected edges, in practice ``thinning'' the edges. The fourth step classifies the edge pixels to three classes separated by empirically determined threshold values. The pixels with gradient value above the high threshold are marked strong pixels and the pixels with gradient value below the low threshold are supressed. In the fifth step the remaining weak pixels with gradient values below the high threshold are preserved or supressed according to the presence of strong pixels in their neighborhood. Detailed description of the algorithm is presented in the original paper by Canny \cite{canny1986computational}, information about implementing a canny edge detector is available in \cite{gonzalez2008digital} and comparison of its performace to other edge detectors can be found in \cite{maini2009study}.
