The actual edge detection in the canny edge detector begins with applying the sobel operator to the input image. The sobel operator is a discrete differentiation operator. It consists of two 3x3 kernels which are convolved with the image to approximate the derivatives. The two kernels represent horizontal and vertical changes. At each point in the image the resulting gradient approximations are combined giving an approximate gradient magnitude. \cite{gonzalez2008digital} The convolution operations are presented in figure \ref{fig:sobelmat}.

\begin{figure}
    \begin{displaymath}
        G_{x} = \begin{bmatrix}
            -1 & 0 & +1 \\
            -2 & 0 & +2 \\
            -1 & 0 & +1 \\
        \end{bmatrix} \ast A
    \end{displaymath}

    \begin{displaymath}
        G_{y} = \begin{bmatrix}
            -1 & -2 & -1 \\
            0 & 0 & 0 \\
            +1 & +2 & +1 \\
        \end{bmatrix} \ast A
    \end{displaymath}
    \caption{The Sobel Kernels used in the application to compute the gradient approximation. The asterisk denotes the convolution operation. \fixme{figure or formula or what?}}
    \label{fig:sobelmat}
\end{figure}
