To understand the behavior and performance of software systems, quantitative data about the system execution is required. The methods for acquiring this data are explained in this section. First an overview to the performance analysis of software systems is provided and then a closer look to analysis through measurements is taken.

\subsection{Analysing Software Systems}
\label{subsec:analysing-software}
The performance analysis of software systems can be split in to three categories, which are analytical modeling, simulation and measurements. Jain \cite{jain1991art} defines analytical modeling as using mathematical models to abstract the key features of the system under study and using these models to make predictions of the system. Models used in simulations are based on mathematical abstractions as well. The key difference between analytical modeling and simulation is the notion of time used in simulation. Analytical modeling solves the system state at a fixed point in time, in contrast to simulators where the system state is computed iteratively at multiple points in time. Analytical modeling and simulation can be used for explorative study of systems that do not yet exist, whereas analysis using measurements cannot be performed unless the real world system under study exists. \cite{jain1991art}

\fixme{add motivation for method selection}

Performance analysis is used for many different purposes. For example performance analysis can be used to help choose the best performing hardware platform for certain application, or to explore different configurations of an application. Successful analysis requires careful experiment design. The execution of a computer program is a complex interaction of hardware and software components and thus the number of parameters to the analysis grows large. The parameters that are varied in the analysis are selected among all of the parameters and they are called factors \cite{jain1991art}. The factor selection requires clear goals for the analysis and a good understanding of the problem space so that the most relevant factors are chosen. The factor selection of the experiments conducted in this thesis is discussed in section \ref{sec:factorselection}


\subsection{Measuring Software Systems}
\label{subsec:measuring-software}
This thesis studies the OpenEM application through measurements. In this subsection a closer look at measuring software systems is provided.

The successful comparison of software systems requires meaningful and reasonably accurate measurements of the systems under study. Measurements are obtained by monitoring the system while it is being subjected to a particular workload \cite{jain1991art}. Often the monitored applications are built for the comparison purpose only and therefore any workload they are subjected to is an approximation of the real world workload that would be processed by their real world application counterparts. These approximate workloads are called synthetic workloads \cite{jain1991art}. The use of synthetic workload gives more control over the test conditions and most importantly makes the experiments repeatable. Workload selection requires care because the synthetic workload needs to mimic its real world counterpart with high accuracy. Performance analysis is often conducted to understand the performance or feasibility of a software component or a system that does not exist yet. In such situations synthetic workloads need to be used out of necessity.  

The software measurement tools are called monitors which can be implemented both in hardware and in software. Monitors are classified to software monitors, hardware monitors, firmware monitors or hybrid monitors depending on the implementation level of the monitor. The implementation level of the monitor affects the level of events that are convenient to measure with it. For example hardware monitors can monitor the state of registers and hardware counters but have difficulties in observing the status of software constructs such as the execution of functions. The software monitors on the other hand can be used to monitor the status of software components but gathering information about the status of the hardware is more difficult and in some cases impossible. \cite{jain1991art} For example it is very complicated to determine whether a memory operation hit a given level of cache or not using software alone but many hardware platforms offer hardware counters to monitor the cache hits and misses.

\fixme{add what exactly is measured}

\subsection{Analysis of the Data}
\label{subsec:data-analysis}
The goal of the performance analysis is to get actionable results about the systems under study. The data obtained from the models or measurements is not in itself enough for making well-grounded decisions. The models and measurements may yield millions of values for the observed variables and the analyst needs to decide how to best represent the data so that the phenomena behind the data are explained \cite{jain1991art}.

Statistical methods are used to analyze the numerical data and expose the causation and correlation between the factors and the results. Often in the literature simple statistical tools such as mean, mode and standard deviations are used to represent the data in only a few numbers. A more thorough look at the statistical tools is provided in \cite{jain1991art}. \fixme{is this all? add a reference for the claim "often in literature"} 

The results of the analysis are often easiest to understand when presented in graphical form. Graphical representations of the data such as histograms, line charts and bar charts are commonly used. These graphs are very generic and used in many fields to present all kinds of data. There are also more domain specific visualizations of data such as the Gantt charts used to represent schedules in computer context and elsewhere. The visualizations of data are designed to be faster to understand than the corresponding numerical views to the same data but they have their limitations. The graphical representations are inaccurate and if they are not carefully prepared they may present a biased view to the real data. Due to these limitations the visualizations should be prepared carefully and the numerical data they are based on should be also made available. The use of visualizations in performance analysis context is explained in \cite{jain1991art}. \fixme{the graphical stuff is further elaborated in XXX (cite tufte here?)}

\fixme{maybe check other sources as well? how about the stuff saarinen uses? there are also more refs in hanhirova thesis}
