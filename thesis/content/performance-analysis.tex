To understand the behavior and performance of software systems, quantitative data about the system execution is needed. The methods for acquiring this data are explained in this chapter. First an overview to the performance analysis of software systems is provided and then a closer look to analysis through measurements is taken.

The performance analysis of software systems can be split in to three categories, which are analytical modeling, simulation and measurements. Jain \cite{jain1991art} defines analytical modeling as using mathematical models to abstract the key features of the system under study and using these models to make predictions of the system. Models used in simulations are based on mathematical abstractions as well. The key difference between analytical modeling and simulation is the notion of time used in simulation. Analytical modeling solves the system state at a fixed point in time, in contrast to simulators where the system state is computed iteratively at multiple points in time. Analytical modeling and simulation can be used for explorative study of systems that do not yet exist, whereas analysis using measurements cannot be performed unless the real world system under study exists. \cite{jain1991art}

\fixme{add motivation for method selection}

Performance analysis is used for many different purposes. For example performance analysis can be used to help choose the best performing hardware platform for certain application, or to explore different configurations of an application. Successful analysis requires careful experiment design. The execution of a computer program is a complex interaction of hardware and software components and thus the number of parameters to the analysis grows large. The parameters that are varied in the analysis are selected among all of the parameters and they are called factors \cite{jain1991art}. The factor selection requires clear goals for the analysis and a good understanding of the problem space so that the most relevant factors are chosen. The factor selection of the experiments conducted in this thesis is discussed in chapter \ref{chapter:experiments}.
