In this thesis the use of OpenEM to provide task management and scheduling in a stream processing application was investigated. The results suggest that OpenEM scales well to the needs of a simple stream processing application and it simplifies the implementation of a multicore application on the DSP platform without operating system.

The workload studied in this thesis was simple and lacked dynamicity of processing graphs common to stream processing applications. To better understand the performance of OpenEM, two future research approaches could be taken. First, studying the performance of OpenEM scheduler with another artificial workload application could still strenghten the case for using OpenEM as the runtime for stream processing. Instead of a simple, linear processing graph like the one used in the experiments of this thesis, a more dynamic graph with optional processing paths could be studied. Studying this kind of graph would show if the the scheduler is capable of hitting the latency and throughput constraints of the application consistently under varying load. Second, a more complex application could be constructed and measured. For example implementing an MPEG decoder using OpenEM as the runtime system would provide a more complete view to the dynamic performance of OpenEM. MPEG decoder would be an interesting workload because it has been used as something of a standard benchmark in dataflow literature and thus there exists many implementations based on different technologies. Having points of comparison for the workload would help put the performance in broader context.

The idea of using DSPs for stream processing has been researched to some extent but the idea has not made it to the mainstream of computing industry. To make the case for DSPs as stream processors a thorough benchmark should be created that would compare the streaming performance versus energy consumed for similar stream processing applications implemented with DSPs, GPUs and CPUs. Making such comparison is not trivial, since to the author's knowledge there does not exist a single widely accepted measure of streaming performance. The question is complicated further by the need for measuring the energy consumption, which itself is a complex task.

Before conducting such research three problems would have to be solved. First, selection of the measured variables. There are papers, which compare floating point operations per second per watt between the platforms, but FLOPS is not necessarily good measurement for streaming performance. For example, for video stream frames per second and latency would provide a better understanding of the streaming performance than FLOPS. Second, accounting of the energy consumption fairly with all measured processing units is not trivial, since they are commonly installed in different forms of devices. CPUs are installed in sockets on motherboards, GPUs are installed on PCI cards or increasingly as part of integrated chips and DSPs are installed in various devices including PCI cards. Comparing the vendor provided energy specifications is one way to deal with this problem, but it might miss how the devices are actually deployed. Third, the workload would have to be implemented in different patterns for each of the platforms making use of the strengths of each platform equally.

Stotzer et al. conduct an experiment in~\cite{stotzer2013openmp} where a comparison between the differnt processing units is conducted in the context of OpenMP, but the comparison does not include energy measurements.

