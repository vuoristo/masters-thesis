In this thesis the use of OpenEM to provide task management and scheduling in a stream processing application was investigated. The results suggest that OpenEM scales well to the needs of a simple stream processing application and it simplifies the implementation of a multicore application on the DSP platform without operating system. The workload studied in this thesis was simple and lacked dynamicity of the processing graph common to stream processing applications. To better understand the performance of OpenEM, a more complex application should be constructed and measured. For example implementing a MPEG decoder using OpenEM as the runtime system would provide interesting view to the dynamic performance of OpenEM. MPEG decoder would be an interesting workload also because has been implemented for multiple platforms including Texas Instruments TMS320C6678 and thus there should be many points of comparison.

The idea of using DSPs for stream processing has been studied to some extent but it does not enjoy much validation from the industry. To make the case for DSPs as stream processors a thorough benchmark should be created that would compare the streaming performance per unit of energy consumed for similar stream processing applications implemented with DSPs, GPUs and CPUs. Stotzer et al. conduct an experiment in~\cite{stotzer2013openmp} where such a comparison is made in the context of OpenMP but it should be interesting to benchmark higher level applications as well.
