This thesis studies the OpenEM application through measurements. In this subsection a closer look at measuring software systems is provided.

The successful comparison of software systems requires meaningful and reasonably accurate measurements of the systems under study. Measurements are obtained by monitoring the system while it is being subjected to a particular workload \cite{jain1991art}. Often the monitored applications are built for the comparison purpose only and therefore any workload they are subjected to is an approximation of the real world workload that would be processed by their real world application counterparts. These approximate workloads are called synthetic workloads \cite{jain1991art}. The use of synthetic workload gives more control over the test conditions and most importantly makes the experiments repeatable. Workload selection requires care because the synthetic workload needs to mimic its real world counterpart with high accuracy. Performance analysis is often conducted to understand the performance or feasibility of a software component or a system that does not exist yet. In such situations synthetic workloads need to be used out of necessity.  

The software measurement tools are called monitors which can be implemented both in hardware and in software. Monitors are classified to software monitors, hardware monitors, firmware monitors or hybrid monitors depending on the implementation level of the monitor. The implementation level of the monitor affects the level of events that are convenient to measure with it. For example hardware monitors can monitor the state of registers and hardware counters but have difficulties in observing the status of software constructs such as the execution of functions. The software monitors on the other hand can be used to monitor the status of software components but gathering information about the status of the hardware is more difficult and in some cases impossible. \cite{jain1991art} For example it is very complicated to determine whether a memory operation hit a given level of cache or not using software alone but many hardware platforms offer hardware counters to monitor the cache hits and misses.

\fixme{add what exactly is measured}
