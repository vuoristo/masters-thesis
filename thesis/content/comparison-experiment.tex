\fixme{ordering in this section is a bit random and it fails to provide a good idea of what to expect.}
In the first experiment the OpenEM and PREESM filter applications were loaded with similar workloads and their execution was measured. The idea of the first experiment is to examine the dynamic scheduling capabilities of the OpenEM scheduler and the overhead of the OpenEM framework in stream processing. The OpenEM scheduler is hardware accelerated, running on a separate processor in the TMS320C6678 chip. The scheduling is explained in detail in chapter \ref{chapter:openem}. To achieve this objective the OpenEM filter application introduced in chapter \ref{chapter:construction} was measured under different loads and compared to a similar application implemented using PREESM. The PREESM application should be considered a baseline, which demonstrates how a statically scheduled application behaves under dynamic workload.

The static schedule of the PREESM application was regenerated between every measurement setup due to the limitations of the code generation in PREESM framework. The specific limitation was that the parameters of the actor model were translated into static memory allocations in the code generator, and manually changing the generated allocations would have been complicated and prone to error. As a result the actors are scheduled slightly differently between each scenario. The schedules differ in the mapping of actors to cores but not in the ordering of the actors. The actor ordering is defined by the dependencies between the actors and availability of computing resources. To demonstrate the effect of static scheduling, the estimated actor timings of the PREESM application were not modified when changing the frame size. \fixme{to demonstrate the effect... is probably not correct here}

In this experiment the applications are loaded with three different workloads and measured. In addition the OpenEM application is measured under the same load but different numbers of available cores. The experiment is explained in the following subsections. In the first subsection the parameters and factors of the experiment are introduced. Second the different measurement setups are described and third the results of the experiment are presented.
