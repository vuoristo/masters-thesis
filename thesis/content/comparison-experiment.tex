\fixme{ordering in this section is a bit random and it fails to provide a good idea of what to expect.}
In the first experiment the OpenEM and PREESM filter applications were loaded with similar workloads and their execution was measured. The idea of the first experiment is to examine the dynamic scheduling capabilities of the OpenEM scheduler and the overhead of the OpenEM framework in stream processing. The OpenEM scheduler is hardware accelerated, running on a separate processor in the TMS320C6678 chip. The scheduling is explained in detail in chapter \ref{chapter:openem}. To achieve this objective the OpenEM filter application introduced in chapter \ref{chapter:construction} was measured under different loads and compared to a similar application implemented using PREESM. The PREESM application should be considered a baseline, which demonstrates how a statically scheduled application behaves under dynamic workload.

The static schedule of the PREESM application was regenerated between every measurement setup due to the limitations of the code generation in PREESM framework. The specific limitation was that the parameters of the actor model were translated into static memory allocations in the code generator, and manually changing the generated allocations would have been complicated and prone to error. As a result the actors are scheduled slightly differently between each scenario. The schedules differ in the mapping of actors to cores but not in the ordering of the actors. The actor ordering is defined by the dependencies between the actors and availability of computing resources. To demonstrate the effect of static scheduling, the estimated actor timings of the PREESM application were not modified when changing the frame size. \fixme{to demonstrate the effect... is probably not correct here}

In this experiment the applications are loaded with three different workloads and measured. In addition the OpenEM application is measured under the same load but different numbers of available cores. The experiment is explained in the following subsections. In the first subsection the parameters and factors of the experiment are introduced. Second the different measurement setups are described and third the results of the experiment are presented.

\subsection{Parameters and Factors}
Dynamic workload conditions are emulated by repeating the measurements with different factors. To keep things simple the video streams are not dynamically switched at runtime. The measurement parameters are presented in the following listing.

\begin{itemize}
    \item \textbf{Video Frame Size} - The workloads are differentiated by changing the frame sizes of the video streams.
    \item \textbf{OpenEM Core Masks} - The OpenEM application is measured with different core masks of the Execution Objects.
    \item \textbf{Number of Frames Processed Simultaneously} - The OpenEM application processes variable number of frames simultaneously.
\end{itemize}

The different video frame sizes used are presented in table \ref{tab:cif_frames}. The frame sizes used are selected from among the Common Intermediate Format frame sizes. \fixme{find a reference for CIF.} YUV video format is used in the applications, but only the Y channel is processed by the applications. The Y channel in the YUV frame contains $R_{x} * R_{y}$ bytes where $R$ is the resolution. In the YUV format used the U and V channels have reduced bitrates of $\frac{1}{4} * R_{x} * R_{y}$ per frame.

\begin{table}
    \begin{center}
        \begin{tabular}{ c c c }
            Name  & X resolution  & Y resolution \\ \hline
            QCIF  & 176           & 144          \\ \hline
            CIF   & 352           & 288          \\ \hline
            4CIF  & 704           & 576          \\ \hline
        \end{tabular}
        \caption{CIF frame sizes}
        \label{tab:cif_frames}
    \end{center}
\end{table}

\fixme{super detached. explain when and how and why} The OpenEM measurements are run with different numbers of frames processed simultaneously to examine the balancing between throughput and latency.

In the second part of this experiment OpenEM core masks are used to limit the number of cores available to the filter application. The behavior of OpenEM is examined under limited resources. The core masks in Texas Instruments implementation of OpenEM are limited so that only one core mask can be active in the application as discussed in chapter \ref{chapter:openem}, therefore the core masks always apply to all execution objects of the application. \fixme{the detail about ti core masks isn't needed here but more info about what is the experiment going to show would be great.}

\subsection{Measurement Setups}
The filter applications process two video streams simultaneously as described in chapter \ref{chapter:construction}. One video stream is processed with a sobel filter and the other is processed with a gaussian filter. The dynamic behavior of the applications is investigated using different workloads. The workloads used are presented in the table \ref{tab:preesm_setups}. The purpose of the different bitrates used for each video stream is to expose the behavior of the OpenEM scheduler in handling dynamic workloads. The static schedule in the PREESM application will provide a baseline to reflect the OpenEM performance to, but again the performance of the applications should not be directly compared due to the difference in the runtime systems.

In addition to comparing the PREESM and OpenEM applications the OpenEM application is measured with different core masks to investigate the dynamic scheduling with different limitations. Both filters of the OpenEM application are loaded with CIF streams and different numbers of cores are used. The experiment is run with core masks allowing one to eight cores being used for processing the streams.

\fixme{how about the initial events?}
\fixme{maybe the ``first experiment'' should be split in to a number of smaller experiments to make more sense out of this chapter}

\begin{table}
    \begin{center}
        \begin{tabular}{ c c }
            Sobel Resolution & Gauss Resolution \\ \hline
            CIF              & CIF              \\ \hline
            4CIF             & CIF              \\ \hline
            CIF              & 4CIF             \\ \hline
            QCIF             & QCIF             \\ \hline
        \end{tabular}
        \caption{PREESM and OpenEM measurement setups}
        \label{tab:preesm_setups}
    \end{center}
\end{table}
