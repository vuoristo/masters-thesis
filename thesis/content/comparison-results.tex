\fixme{Results and discussion on results are both kind of formless blobs of text commenting on mixed findings from the experiments. calls for better division of topics and more structure.}

In this subsection first the results of the measurements of the OpenEM and PREESM applications are presented and next the results of measuring the OpenEM application limited with core masks are presented. The latency and throughput of the OpenEM application measurements are summarized in the tables \ref{tab:oemthrough} and \ref{tab:oemthrough2}. The corresponding summaries for the PREESM application are presented in the table \ref{tab:preesmthrough} The summary of latency and throughtput of the OpenEM application with limited number of cores is presented in table \ref{tab:oemcoremasks}.

The latency of both of the filters is measured from the time the frame is loaded from the memory to the time the frame is merged after filtering. The throughput is measured as frames per second processed in total by the application. Since both of the applications process frames at the same rate from both streams, there is only one FPS measurement for each of the measurement setups. \fixme{maybe move this to construction chapter?}

The latencies of the two streams of the PREESM application in all measurement setups in the table \ref{tab:preesmthrough} are similar. The largest difference in the latencies is in the case where sobel filter filters a CIF stream and gauss filter filters a 4CIF stream, where gauss latency is 60\% higher than the sobel latency. Much larger differences in the latencies are observed with the OpenEM application with two initial events presented in table \ref{tab:oemthrough2} where the largest difference in latencies is also measured between the CIF sobel stream and 4CIF gauss stream. The gauss latency in the particular OpenEM measurement is 200\% higher than the sobel latency. The other OpenEM measurements show larger differences as well compared to PREESM.

\newcommand{\head}[2]{\multicolumn{1}{>{\centering\arraybackslash}p{#1}}{#2}}
\begin{table}
    \begin{center}
        \begin{tabular}{ c c c c c }
            \head{1.5cm}{Sobel latency} & \head{1.5cm}{Gauss latency} &
            \head{1.5cm}{FPS} & \head{1.5cm}{Sobel frame} &
            \head{1.5cm}{Gauss frame} \\ \hline
            5,41 & 8,78 & 223 & CIF & 4CIF \\ \hline
            4,65 & 3,54 & 334 & 4CIF & CIF \\ \hline
            2,15 & 2,51 & 668 & CIF & CIF \\ \hline
            0,61 & 0,71 & 2004 & QCIF & QCIF \\ \hline
        \end{tabular}
        \caption{PREESM latency and throughput. The the latencies are measured in milliseconds.}
        \label{tab:preesmthrough}
    \end{center}
\end{table}
\begin{table}
    \begin{center}
        \begin{tabular}{ c c c c c }
            \head{1.5cm}{Sobel latency} & \head{1.5cm}{Gauss latency} &
            \head{1.5cm}{FPS} & \head{1.5cm}{Sobel frame} &
            \head{1.5cm}{Gauss frame} \\ \hline
            3,59 & 10,93 & 256 & CIF & 4CIF \\ \hline
            3,75 & 2,95 & 527 & 4CIF & CIF \\ \hline
            1,42 & 2,80 & 889 & CIF & CIF \\ \hline
            0,32 & 0,71 & 3534 & QCIF & QCIF \\ \hline
        \end{tabular}
        \caption{OpenEM latency and throughput with 2 frames processed simultaneously. The latencies are measured in milliseconds.}
        \label{tab:oemthrough2}
    \end{center}
\end{table}
\begin{table}
    \begin{center}
        \begin{tabular}{ c c c c c }
            \head{1.5cm}{Sobel latency} & \head{1.5cm}{Gauss latency} &
            \head{1.5cm}{FPS} & \head{1.5cm}{Sobel frame} &
            \head{1.5cm}{Gauss frame} \\ \hline
            15,82 & 22,85 & 599 & CIF & 4CIF \\ \hline
            4,85 & 3,67 & 895 & 4CIF & CIF \\ \hline
            4,91 & 5,96 & 1955 & CIF & CIF \\ \hline
            1,33 & 1,62 & 7819 & QCIF & QCIF \\ \hline
        \end{tabular}
        \caption{OpenEM latency and throughput with 16 frames processed simultaneously. The latencies are measured in milliseconds.}
        \label{tab:oemthrough}
    \end{center}
\end{table}

The PREESM latencies in the table \ref{tab:preesmthrough} are consistently smaller than the latencies of the throughput optimized OpenEM application in the table \ref{tab:oemthrough}. The throughput versus latency balance in the OpenEM application is controlled through the number of frames processed simultaneously.  In this set of measurements the OpenEM application is configured to process 16 frames simultaneously to maximize throughput. \fixme{why is this discussed only here? how about moving this info to construction?} The effect of increasing the number of simultaneous frames was examined by measuring the application with two CIF streams using 2 to 24 initial events. The results of the measurements are presented in table \ref{tab:oeminitialframes}. The latencies get worse and the throughput gets better when the number of frames processed simultaneously increases. This behavior is visualized in figures \ref{fig:oeminitialframesfps} and \ref{fig:oeminitialframeslat}. The throughput grows rapidly up to 8 simultaneous frames after which the growth slows down. The latencies grow with each added frame steadily.

\begin{figure}
    \centering
    \begin{subfigure}[t]{0.49\textwidth}
        \centering
        \includegraphics[width=0.99\linewidth]{images/simultaneous_frames_fps.eps}
        \caption{FPS as a function of simultaneous frames.}
        \label{fig:oeminitialframesfps}
    \end{subfigure}
    \begin{subfigure}[t]{0.49\textwidth}
        \centering
        \includegraphics[width=0.99\linewidth]{images/simultaneous_frames_latency.eps}
        \caption{Latencies as functions of simultaneous frames. The latencies are measured in millisecods.}
        \label{fig:oeminitialframeslat}
    \end{subfigure}
    \caption{The effect of increasing the number of frames processed simultaneously is presented in the figures. The throughput increases rapdily up to 8 simultaneous frames after which the growth slows down. The latency grows more steadily across all measured setups.}
\end{figure}

The throughput of the application is determined by how much time the application spends processing the streams versus the time spent doing something else. For example the synchronization of all cores before each repetition of the PREESM schedule consumes a lot of cpu cycles as is readily observed from the figure \ref{fig:preesmcif}. The portion of the bars marked as busy corresponds to the cycles spent in the synchronization between the repetitions of the schedule. The percentage of cycles spent in the synchronization varies from 16\% on Core 7 to 45\% on Core 0. The OpenEM dynamic scheduler seems to spread out the work more evenly in this case where the total overhead cycles are approximately 60\% for all cores. The core utilization per function is presented in figure \ref{fig:oem8corefunc}. Most of the overhead cycles in the OpenEM application are spent waiting for more frames for processing.

\begin{figure}
    \centering
    \begin{subfigure}[t]{0.49\textwidth}
        \centering
        \includegraphics[width=0.99\linewidth]{images/preesm_cifcif.eps}
        \caption{PREESM}
        \label{fig:preesmcif}
    \end{subfigure}
    \begin{subfigure}[t]{0.49\textwidth}
        \centering
        \includegraphics[width=0.99\linewidth]{images/openem_cifcif_2initial_func.eps}
        \caption{OpenEM}
        \label{fig:oem8corefunc}
    \end{subfigure}
    \caption{In the PREESM graph the busy portion of the bars correspond to the cycles spent synchronizing the cores between the repetitions of the block schedule. The overhead corresponds to cycles spent outside the measured functions and the synchronization. In the OpenEM application the overhead cycles are the cycles spent outside the compared functions.}
\end{figure}

The overhead portions in the figures \ref{fig:preesmcif} and \ref{fig:oem8corefunc} contain all of the data copying from buffer to buffer outside the measured functions, but they also contain different amounts of cycles spent in communications between the cores. The measured functions are the functions which are explicitly called in the PREESM actor model. Other functions not included in the measured functions contain runtime specific communication and some copying of buffers.

\begin{table}
    \begin{center}
        \begin{tabular}{ c c c c }
            \head{1.5cm}{Sobel latency} & \head{1.5cm}{Gauss latency} &
            \head{1.5cm}{FPS} & \head{1.5cm}{Number of cores} \\
            \hline
            57,05 & 57,11 & 263 & 1 \\ \hline
            22,59 & 23,15 & 510 & 2 \\ \hline
            15,09 & 15,84 & 768 & 3 \\ \hline
            10,91 & 11,85 & 1014 & 4 \\ \hline
            8,41 & 9,53 & 1268 & 5 \\ \hline
            7,05 & 8,07 & 1500 & 6 \\ \hline
            5,74 & 6,83 & 1731 & 7 \\ \hline
            4,91 & 5,96 & 1955 & 8 \\ \hline
        \end{tabular}
        \caption{OpenEM measurements with number of cores varied}
        \label{tab:oemcoremasks}
    \end{center}
\end{table}

The second part of this experiment was to limit the number of cores available for the OpenEM application. The resulting latencies and throughputs are presented in the table \ref{tab:oemcoremasks}. The increase of the throughput of the OpenEM application is presented in graph \ref{fig:fpsvcores}. The actual FPS measured with eight cores is approximately 90\% of the linear growth.  

\begin{figure}[h!]
    \begin{center}
        \includegraphics[width=0.7\textwidth]{images/coremask_fps.eps}
        \caption{FPS increase as the function of cores vs. linear growth}
        \label{fig:fpsvcores}
    \end{center}
\end{figure}

\begin{table}
    \begin{center}
        \begin{tabular}{ c c c c }
            \head{1.5cm}{Sobel latency} & \head{1.5cm}{Gauss latency} &
            \head{1.5cm}{FPS} & \head{1.5cm}{Simultaneous Frames} \\
            \hline
            1,42  &  2,80  &  889   &  2 \\ \hline
            1,72  &  3,22  &  1371  &  4 \\ \hline
            2,26  &  3,59  &  1655  &  6 \\ \hline
            2,67  &  3,96  &  1825  &  8 \\ \hline
            3,13  &  4,37  &  1880  &  10 \\ \hline
            3,68  &  4,89  &  1921  &  12 \\ \hline
            4,27  &  5,40  &  1940  &  14 \\ \hline
            4,73  &  5,88  &  1958  &  16 \\ \hline
            5,43  &  6,52  &  1976  &  18 \\ \hline
            6,22  &  7,21  &  1983  &  20 \\ \hline
            6,83  &  7,80  &  1981  &  22 \\ \hline
            7,41  &  8,45  &  1992  &  24 \\ \hline
        \end{tabular}
        \caption{OpenEM measurements with different numbers of frames in processed simultaneously.}
        \label{tab:oeminitialframes}
    \end{center}
\end{table}
