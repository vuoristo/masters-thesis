Task / Event parallelism

Task is a rather ambiguous term that means a unit of execution or a unit of work. Task parallelism is a form of parallelism, which focuses on distributing tasks among multiple cores. Task parallelism does not define the form of tasks distributed. The concrete tasks are executed by threads or processess distributed across cores.

Concurrency is more and more common in modern computing. There are multiple reasons why concurrency is so important in programming. Probably the biggest reason for increased concurrency in software is that the single-threaded performance of the hardware is not growing as rapidly as it was in the past~\cite{sutter2005free}. Apart from increasing the amount of computation done in parallel, other reasons for increasing concurrency in programs are handling asynchronous IO and responding to external events such as user interactions.

IO operations often take long compared to computation on the cpus. In IO heavy applications multiple IO operations can be started concurrently and their results processed in the order of completion. Such concurrent waiting increases the cpu utilization as the cpus are not spending time busy waiting for IO but instead processing the results of completed IO operations.~\cite{dabek2002event}

A common way of enabling concurrency in programs is to split the execution of the program across multiple software threads. Using threads for concurrent programming is convenient because they allow interleaving I/O and computation while preserving the appearance of a serial program~\cite{dabek2002event}. The disadvantage of threads is that they introduce concurrency even to sections of programs where it is not needed. Programming with threads requires explicit synchronization of the threads, which in practice yields data races and deadlocks.~\cite{dabek2002event}

\cite{dabek2002event} argues for events instead of threads to provide concurrency in IO heavy server environments. The benefits of events are that they provide the same benefits as threads in concurrent io programs but are easier to program and tend to yield more stable performance under heavy loads.

Motivate event-driven programming.

Getting the terminology in order:

event-driven programming? or event-driven processing?
task vs. event?

define event

Define the event driven X.

Event driven vs. threads


Event loop
What is an event loop

Examples of event driven systems
SCnC?
Cilk
Node.js
Eve \cite{fonseca2014eve}
OpenMP Streaming Extensions? \cite{pop2011stream}

What is the distinction of Event driven processing vs. events in GUI programming? Node.js seems to combine both as its processing model is inherited from the web frontend world but it is used for various tasks on the server side. Also wikipedia article for node.js states that node.js is specifically event-driven.

OpenEM states it is an architectural abstraction and framework of an event driven multicore optimized processing concept

C++ standard does not define events. The event handling is left to be provided by frameworks which generate the events.

Task / Event. <- focus on these
  Task parallelism
    computation divided into tasks
    explicit parallelism
    no synchronization required inside tasks
  Input is converted to tasks
  Task based processing does not require underlying data streams

Event based 
  Task based MoC with the concept of stream included
  can this jump be made?

SCnC
  Concurrent collection with stream

Eve
  Programming language

OpenMP streaming extension

In general events do not need to be io events
  In this thesis they are io events

Event lib
node.js

