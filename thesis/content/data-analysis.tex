The goal of the performance analysis is to get actionable results about the systems under study. The data obtained from the models or measurements is not in itself enough for making well-grounded decisions. The models and measurements may yield millions of values for the observed variables and the analyst needs to decide how to best represent the data so that the phenomena behind the data are explained \cite{jain1991art}.

Statistical methods are used to analyze the numerical data and expose the causation and correlation between the factors and the results. Often in the literature simple statistical tools such as mean, mode and standard deviations are used to represent the data in only a few numbers. A more thorough look at the statistical tools is provided in \cite{jain1991art}. \fixme{is this all? add a reference for the claim "often in literature"} 

The results of the analysis are often easiest to understand when presented in graphical form. Graphical representations of the data such as histograms, line charts and bar charts are commonly used. These graphs are very generic and used in many fields to present all kinds of data. There are also more domain specific visualizations of data such as the Gantt charts used to represent schedules in computer context and elsewhere. The visualizations of data are designed to be faster to understand than the corresponding numerical views to the same data but they have their limitations. The graphical representations are inaccurate and if they are not carefully prepared they may present a biased view to the real data. Due to these limitations the visualizations should be prepared carefully and the numerical data they are based on should be also made available. The use of visualizations in performance analysis context is explained in \cite{jain1991art}. \fixme{the graphical stuff is further elaborated in XXX (cite tufte here?)}

\fixme{maybe check other sources as well? how about the stuff saarinen uses? there are also more refs in hanhirova thesis}
