In this thesis an experiment on implementing and analyzing a stream processing program is conducted. The program is implemented on Texas Instruments (TI) hardware using Open Event Machine (OpenEM). The program will be implemented in a dataflow pattern.

In OpenEM the general problem of locking in concurrent programming is transformed in to a scheduling problem. Whenever two threads need the same resource they are scheduled sequentially so that they don't try access the resource simultaneously.

When OpenEM schedules an event, it allocates the resources for that event and lets it run to completion. The next event utilizing the same resources might be something completely different depending on which events are ready for execution. There are no local buffers tied to certain memory locations, as all resource allocation and event execution is scheduled on the fly. This thesis studies the memory behavior in such dataflow context, through the application implemented and the analysis conducted on it. 

The execution of the program will be statically analyzed using a suitable tool. The current candidate for analysing the execution is implemented in a tracing library provided by Texas Instruments.

The execution will be dynamically analyzed by measuring the real system. In addition a simulation model of the system will be constructed. The execution of the simulation model will be monitored. Results from the simulation model and the real system will be compared.

\subsection{Construction}
An actor based implementation of Sobel filter will be used as base for the workload construction. This implementation will be rebuilt on top of OpenEM preserving the actors but changing the details to match the runtime. The structure of the application construction is presented in the following graph.

\begin{samepage}
\begin{verbatim}
**************************************************
|     Video processing workload in Actor Model   |
**************************************************
 . . . . . . . . . . OpenEM . . . . . . . . . . .
**************************************************
|           Texas Instruments Hardware           |
**************************************************
\end{verbatim}
\end{samepage}

Another filter such as Canny will be constructed following a similar pattern. The initial actor based implementation will be constructed in PREESM. The PREESM implementation will be converted to OpenEM.

The two (or more) filters are used to simultaneously process two or more streams on the same DSP. The simple implementations (Actors, PREESM) will be scheduled with hand derived static schedule. The benchmarking setup will be constructed so that the number of streams can be changed dynamically.

The performance of the OpenEM based version will be compared to the statically scheduled version under different load conditions. Dynamically changing numbers of streams are especially interesting for the goals of this thesis. Static loads will be studied to find out the overhead of dynamic scheduling in OpenEM versus the static scheduling. 

\subsection{Analysis}
Before choosing the exact analysis tools to be used, more information needs to be gathered about the tools and the workload to be analyzed. There is some internal interest in using the PSE tool for dynamic analysis and since Jussi Hanhirova has been exploring the tool and the using it to analyze applications for TMS320C6678 it seems like a logical choice on the Dynamic analysis side.

Before execution: static analysis <- WCET (Actor Accelerator Function)

After Execution: dynamic analysis <- scheduling (PSE)
