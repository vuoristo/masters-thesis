A realistic workload application with capability for processing dynamic input is designed for the experiment. DSPs are commonly used for video stream processing. Processing video streams is a suitable workload for parallelization and dynamic number of inputs. Video filtering is a common and fairly simple task in video stream processing. For these reasons a video filtering application is selected as the workload for the experiment.

The workload consists of two common filters. The first is a sobel filter that is used in edge detection. The second is a gaussian filter that has a wide range of uses. Both filters are used in a canny edge detector which is a realistic workload for a multicore DSP. The real world users of the studied device would probably implement a complete edge detector rather than parts of it. However the workload in this experiment is deliberately kept simple to make it well analyzable and reasonably simple to implement.

The filters themselves are based on convolving the image data with a filter matrix and are thus straightforward to implement.

\textbf{Gaussian Filter} http://en.wikipedia.org/wiki/Gaussian\_filter\\
\textbf{Sobel Operator} http://en.wikipedia.org/wiki/Sobel\_operator\\
\textbf{Canny Edge Detector} http://en.wikipedia.org/wiki/Canny\_edge\_detector\\