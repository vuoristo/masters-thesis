\subsection{Open Event Machine}
Open Event Machine is a runtime system for multicore platforms originally developed by NSN for network dataplane. OpenEM is used to write dynamically load balanced applications with run-to-completion principle.

The key concepts of OpenEM are events, execution objects, queues and the scheduler. \textbf{Event} is the unit of communication in OpenEM. Events are typically used to carry data to be processed but can be data-less tokens as well. The event descriptors are allocated at initialization of the Queues. \textbf{Execution objects} encapsulate the algorithm to execute when an event is received. \textbf{Queues} connect events (data) and execution objects (algorithms). Each queue is associated with one execution object and all queued events will be processed by this execution object. \textbf{Scheduler} moves allocated events to queues. \textbf{Dispatcher} is called by the user in a dispatch loop. Dispatcher calls the ‘receive’ function of the Execution object of the connected queue.

\subsection{Texas Instruments Implementation of OpenEM}
The Texas Instruments implementation of Open Event Machine runtime system is included in the MCSDK (chapter \ref{MCSDK}). OpenEM version 1.0.0.2 is used in the experiment. The OpenEM library is OS agnostic: the OpenEM dispatcher can be called from bare metal software or from an OS thread. The TI implementation of OpenEM Leverages the Multicore Navigator for hardware queues and packet communication. The scheduler always runs on PDSP core and is defined by OpenEM specific firmware.

The OpenEM execution model is described in figure \ref{tiem}. The event descriptors are allocated at OpenEM initialization in the free pool. The application allocates events from the free pool using \textit{em\_alloc}. After the allocated event has been populated it is sent to a queue using \textit{em\_send}. The queues are associated with specific execution objects at initialization and therefore they strictly determine which EO will process the events in any of the queues. The scheduler running on a PDSP core (chapter \ref{navigator}) manages the queues (virtual queues mapped to QMSS hardware queues). The scheduling operations are carried out on an explicit request from a DSP core when a call to \textit{ti\_em\_preschedule} is made. When one of the cores mapped to specific execution object finished executing its current event it makes a call to \textit{ti\_em\_dispatch\_once} which triggers a call to the \textit{receive} function of the execution object. The execution object processes the data and makes a call to \textit{em\_free} to return the event to its free pool.

\begin{figure}[h]
\begin{center}
\includegraphics[width=1.3\textwidth,natwidth=701,natheight=609]{openem_model.png}
\caption{The OpenEM execution model}\label{tiem}
\end{center}
\end{figure}

\textbf{Texas Instrument} OpenEM White Paper
\textbf{Texas Instrument} OpenEM User Guide


\subsection{Video Filter Application Using OpenEM}