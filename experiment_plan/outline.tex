\subsection{Outline}
This document describes the construction of an OpenEM dynamic workload experiment. OpenEM implementation on Texas Instruments TMS320C6678 will be used as the runtime system. The workload will be a simple video filtering application consisting of couple of filters and dynamic number of streams.

The implementation steps of the experiment are represented in the following figure.

\begin{itemize}
\item[] \textbf{Iteration 1}
    \subitem Filter application in PREESM
    \subitem Convert Filter application to OpenEM
    \subitem Implement tracing in both applications
    \subitem Model the OpenEM application with PSE
    \subitem Select a suitable workload for the applications
    \subitem Measure applications
    \subitem Compare measurements with the PSE estimates
    \subitem Check fitness of model to application behavior
\item[] \textbf{Iteration 2}
    \subitem Compare the model behavior to expected OpenEM behavior
    \subitem If the model behavior matches expected OpenEM behavior but not the real application behavior study the cause
    \subitem Does the application utilize OpenEM according to specification?
    \subitem Does OpenEM specification match the implementation on the platform
    \subitem If faults in either model or the application are found fix them
\end{itemize}

\subsection{Goals}
The objective of this experiment is to understand the behavior and performance of Texas Instruments implementation of Open Event Machine in processing a dynamic workload. The performance of an application implemented using OpenEM will be compared to the performance of a similar application implemented using a simpler multicore runtime. Open Event Machine utilizes a hardware accelerated packet communication system called The Multicore Navigator. The simpler runtime doesn't utilize hardware acceleration. The focus of this experiment is in understanding the effects of hardware accelerated scheduling vs. simple scheduling.

The OpenEM implementation of the workload application will be modeled using Performance Simulation Environment. The PSE simulation results are compared to the performance of the real world application. The simulation model created in PSE and validated on one real world application could be used to estimate the performance of similar applications on the hardware platform studied in this experiment or the performance of the application on similar hardware platforms.